\section{A Concurrent Stack with Helping}

The abstract data type that we're implementing is that of a
stack. Therefore, it will have two operations, {\tt push} and
{\tt pop}. The main complication of our data structure is that {\tt
  push} and {\tt pop} must be thread-safe.

In order to handle this in the implementation, we will make use of the
fact that we have an atomic compare-and-swap operation on
reference cells in our language. By storing our stack in such a cell,
we can retrieve it, modify it as necessary, and attempt to atomically
replace the old stack with our new stack. As is usual with
fine-grained concurrency, it is entirely possible that halfway
through our operation another thread might modify the stack. In this
case, the attempt to replace the old stack with the new stack using a
CAS instruction will fail. The typical way to handle this is to loop
and attempt the operation from the beginning, though in high
contention situations this can be a major delay and cause long running
operations to live-lock.

In order to partially mitigate this, it would be ideal if some of our
concurrent operations could avoid dealing with the main cell holding
the stack entirely. For example, if one thread is attempting a push at
the same time another is attempting a pop, they will fight each other
for ownership of the cell even though it would be perfectly valid for
one thread to simply hand the other the value. In a high contention
situation with many threads pushing and popping at once, this can be
quite a common problem. We handle this by introducing a
\emph{side-channel} for threads to communicate along. That is, before
a thread attempts to work with the main stack it will, for instance,
check whether or not someone is offering a value along the
side-channel that it could just take instead. Similarly a thread which
is pushing a value onto the stack will offer its value on the
side-channel temporarily in an attempt to avoid having to compete for
the main stack. This scheme reduces contention on the main atomic cell
and thus improves performance.

\subsection{Mailboxes for Offers}
In order to do this, before designing a stack we first implement a
small API for side-channels. A side-channel has the following
operations
\begin{enumerate}
\item An offer can be \emph{created} with an initial value.
\item An offer can be accepted, marking the offer as taken and
  returning the underlying value.
\item Once created, an offer can be revoked which will prevent anyone
  from accepting the offer and return the underlying value to the
  thread.
\end{enumerate}
Of course, all of these operations have to be thread-safe! That is, it
must be safe for an offer to be attempted to be accepted by multiple
threads at once, an offer needs to be able to be revoked while it's
being accepted, and so on. We choose to represent an offer as a tuple
of the actual value the offer contains and a reference to an int. The
underlying integer may take one of 3 values, either $0$, $1$ or
$2$. Therefore, an offer of the form $(v, \ell)$ with $\ell \mapsto 0$
is the initial state of an offer, no one has attempted to take it nor
has it been revoked. Someone may attempt to take the offer in which
case they will use a CAS to switch $\ell$ from $0$ to $1$, leading to
the accepted state of an offer which is $(v, \ell)$ so that
$\ell \mapsto 1$.  Revoking is almost identical but instead of
switching from $0$ to $1$ instead we switch to $2$. Since both
revoking and accepting an offer demand the offer to be in the initial
state it is impossible anything other than exactly one accept or one
revoke to succeed. The actual code for this is in
Figure~\ref{fig:code:offer} and the transition graph sketched above is
illustrated in Figure~\ref{fig:code:offertransgraph}.
\begin{figure}
\begin{lstlisting}
  mk_offer = fun v -> (v, ref 0)
  revoke_offer =
    fun v ->
      if cas (snd v) 0 2
      then Some (fst v)
      else None
  accept_offer =
    fun v ->
      if cas (snd v) 0 1
      then Some (fst v)
      else None
\end{lstlisting}
\caption{The implementation of offers used to construct side-channels.}
\label{fig:code:offer}
\end{figure}
\begin{figure}
  \begin{tikzpicture}[->,>=stealth',shorten >=1pt,auto,node distance=4cm,
    thick,main node/.style={circle,draw,font=\sffamily\Large\bfseries}]

    \node[main node] (1) at (0, 0)    {(-, 0)};
    \node[main node] (2) at (4,  1.0) {(-, 1)};
    \node[main node] (3) at (4, -1.0) {(-, 2)};
    \path[every node/.style={font=\sffamily\small}]
    (1) edge[bend left]  node[pos=0.8, text depth=0.4cm,  left] {accept} (2)
        edge[bend right] node[pos=0.8, text depth=-0.4cm, left] {revoke} (3);
  \end{tikzpicture}
  \caption{The states an offer may be in.}
  \label{fig:code:offertransgraph}
\end{figure}
The pattern of offering something, immediately revoking it, and
returning the value if the revoke was successful is sufficiently
common that we can encapsulate it in an abstraction called a
\emph{mailbox}. The idea is that a mailbox is built around an
underlying cell containing an offer and that it provides two functions
which, respectively, briefly put a new offer out and check for such an
offer. The code for this may be found in
Figure~\ref{fig:code:mailbox}. One small technical detail is that we
have designed this as a constructor which returns two closures which
manipulate the same reference cell. This simplifies the process of
using a mailbox for stacks where we only have one mailbox at a
time. It is not, however, fundamentally different than an
implementation more in the style of the offers above.
\begin{figure}
\begin{lstlisting}
  mailbox = fun () ->
    let r = ref None in
    (rec put v ->
       let off = mk_offer v in
       r := Some off;
       revoke_offer off,
     rec get n ->
       let offopt = !r in
       match offopt with
         None -> None
       | Some x -> accept_offer x
       end)
\end{lstlisting}
\caption{The implementation of mailboxes which provide a convenient
  wrapper over offers.}
\label{fig:code:mailbox}
\end{figure}
\subsection{The Implementation of the Stack}
With an implementation of offers, it is easy to code up the concurrent
stack. The idea for implementing a side-channel is to have a
designated cell for threads to put pending offers in. This way, when a
thread comes along to push a value onto the stack it will first create
an offer using the above API and put it into the given cell. It will
then immediately revoke it to see if another thread has accepted the
offer in the meantime and if none has, it will proceed with the
pushing algorithm. The dual process suffices for a thread seeking to
pop. The code for the stack is in Figure~\ref{fig:code:stack}. Notice
that this too is written in a similar style to that of mailboxes, a
make function which returns two closures for the operations rather
than having them separately accept a stack as an argument.
\begin{figure}
\begin{lstlisting}
  stack = fun () ->
    let mailbox = mailbox () in
    let put = fst mailbox in
    let get = snd mailbox in
    let r = ref None in
    (rec pop n ->
       match get () with
         None ->
         (match !r with
            None -> None
          | Some hd =>
            if cas r (Some hd) (snd hd)
            then Some (fst hd)
            else pop n
          end)
       | Some x -> Some x
       end,
      rec push n ->
        match put n with
          None -> ()
        | Some n ->
          let r' = !r in
          let r'' = Some (n, r') in
          if cas r r' r''
          then ()
          else push n
        end)
\end{lstlisting}
\caption{The implementation of the concurrent stack.}
\label{fig:code:stack}
\end{figure}

%%% Local Variables:
%%% mode: latex
%%% TeX-master: "main"
%%% End:
