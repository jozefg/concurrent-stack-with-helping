\section{A Third Formalization: View-Shifts with Helping}

It is a straightforward exercise to adapt the previous proof to work
with helping using the same setup as the previous proof. Indeed, since
helping is an invisible optimization to a concurrent data structure
the same specification ought to apply. The only question is how to
modify the specifications we give to mailboxes in order to suitably
handle these view-shifts.

The primary difference is in the definition of the invariants for
offers and mailboxes. Rather than having them pass around ownership of
$P(v)$ for some $P$ where $v$ is the value they contain, they have to
pass around the right to perform certain view-shifts. This
unfortunately means that the proofs for mailboxes and offers become
entangled with that of stacks to a much larger degree. This is why the
Coq development does not isolate them into separate lemmas any
more.

The new invariant for offers is now defined using
\[
  \textlog{stages}_\gamma(\ell, v) \triangleq
  (\ell \mapsto 0 * (\All vs. P(vs) \vsW P (v :: vs) * Q))
  \mathrel{\vee} (\ell \mapsto 1 * Q)
  \mathrel{\vee} (\ell \mapsto 1 * \ownGhost{\gamma}{\exinj(())})
  \mathrel{\vee} (\ell \mapsto 2 * \ownGhost{\gamma}{\exinj(())})
\]
This means that either one of two interactions is possible
\begin{enumerate}
\item A thread without ownership of $\ownGhost{\gamma}{\exinj(())}$
  attempts to switch from $\ell \mapsto 0$ to $\ell \mapsto 1$. In doing so,
  it is obligated to take the view-shift
  $\All vs. P(vs) \vsW (v :: vs) * Q)$ and place the resulting $Q$
  back in the invariant.
\item The thread which owns $\ownGhost{\gamma}{\exinj(())}$ may
  attempt to revoke this offer, switching it from $\ell \mapsto 0$ to
  $\ell \mapsto 2$. If it is successful then it can return the
  view-shift transitioning $\All vs. P(vs) \vsW (v :: vs) * Q)$ or it
  fails, implying that some thread has already successfully executed a
  take. If the latter is the case, the thread can return the $Q$ that
  must be stored in $\textlog{stages}$ which must be in the disjunct
  $\ell \mapsto 1 * Q$.
\end{enumerate}
With this we can specify the invariant enforced on offers.
\[
  \textlog{is\_offer}_\gamma(\iota, v) \triangleq
  \Exists v' l. v = (v', l) * \Exists \iota'. \iota \disjoint \iota' *
  \knowInv{\iota'}{\textlog{stages}_\gamma(l)}
\]
The only difference worth noting is the inclusion of this
$\iota' \disjoint \iota$. This is purely an artifact of the particular
approach the proof of the specification has taken. At various points,
it will be necessary to simultaneously open the invariant containing
knowledge of the stack as well as the invariant containing what stage
an offer is in. For instance, this is how {\tt take\_offer} will work
so that it can actually apply the view-shift to the current state of
the stack. Opening two invariants at the same time, however, is only
possible if the invariants are in fact disjoint. All that this premise
does is manually record the fact that this is the case. Later on the
$\iota$ parameterizing $\textlog{is\_offer}$ will be the same as the
$\iota$ governing the stack invariant.

This change also propagates through the $\textlog{is\_mailbox}$
invariant since this is after all where $\textlog{is\_offer}$
occurs. Apart from this plumbing though the invariant is unchanged
from the prior section.
\[
  \textlog{is\_mailbox}(\iota, v) \triangleq
  \Exists \ell. v = \ell *
  \ell \mapsto \textlog{None} \mathrel{\vee}
  \Exists v' \gamma. l \mapsto \textlog{Some}(v') * \textlog{is\_offer}_\gamma(\iota, v')
\]
Rather than proceeding to formalize the specifications of mailboxes
and offers, it is simpler to proceed directly to formalizing the
stack data structure itself. The invariant governing a stack is
unchanged from the previous section.
\begin{align*}
  \textlog{is\_stack}([], v) &\triangleq v = \textlog{None}\\
  \textlog{is\_stack}(x :: L, v) &\triangleq v =
  \Exists t. v = \textlog{Some}(x, t) * \textlog{is\_stack}(L, t)\\
  \textlog{stack\_inv}(v) &\triangleq
  \Exists \ell, v', L. v = \ell * \ell \mapsto v' * P(L) * \textlog{is\_stack}(L, v')
\end{align*}
The final specification is again unchanged. For brevity, the proof
(which is largely a combination of the two previously explained ones)
is only sketched.
\begin{thm}
  The following specification holds.
  \begin{align*}
    \forall \Phi.\qquad&\\
    (\All f_1 f_2.& \\
    (&(\All vs. P(v :: vs) \vsW Q_1(v) * P(vs)) \wand \\
                       &(\All vs. P([]) \vsW Q_2 * P([])) \wand\\
                       &\wpre{f_1()}{v. v = \textlog{None} * Q_1 \mathrel{\vee} \Exists v'. v = \textlog{Some}(v') * Q_1(v')}\\
    \wand \quad (&\All v. (\All vs. P(vs) \vsW Q * P(v :: vs)) \wand \wpre{\text{\tt f2(v)}}{v. Q}) \\
    \wand \quad \phantom{(}& \Phi(f_1, f_2))\\
    \wand P([&]) \\
    \wand \wpre{&\textlog{stack}()}{\Phi}
  \end{align*}
\end{thm}
\begin{proof}
  The beginning of the proof is a straightforward combination of the
  previous two proofs, allocating an invariant for the mailbox and the
  mutable cell containing the stack: $\knowInv{\iota}{\textlog{stack\_inv}(r)}$
  and $\knowInv{\iota'}{\textlog{is\_mailbox}(r', \iota)}$. Notice that we pass
  the $\iota$ from the stack invariant to the mailbox invariant as
  discussed above.

  Having done this, we apply the assumption that
  \begin{align*}
    \All f_1, f_2.&\\
    (&(\All vs. P(v :: vs) \vsW Q_1(v) * P(vs)) \wand \\
     &(\All vs. P([]) \vsW Q_2 * P([])) \wand\\
     &\wpre{f_1()}{v. v = \textlog{None} * Q_1 \mathrel{\vee} \Exists v'. v = \textlog{Some}(v') * Q_1(v')})\\
    \wand \quad (&\All v. (\All vs. P(vs) \vsW Q * P(v :: vs)) \wand \wpre{f_2(v)}{v. Q})\\
    \wand \quad \phantom{(}& \Phi(f_1, f_2)
  \end{align*}
  This leaves us with two separate verifications for push and
  pop. These are largely the same as the previous proof. The only
  difference is each proof is proceeded by manual manipulations of
  the mailboxes and offers which in turn are simply inlined versions
  of the proof from the prior sections.

  We will consider the case for {\tt push}. The goal we must prove is
  then
  \[
    \All v. (\All vs. P(vs) \vsW Q * P(v :: vs))
    \wand \wpre{\text{\tt push(v)}}{v. Q}
  \]
  Therefore, let us suppose that $\All vs. P(vs) \vsW Q * P(v :: vs)$
  and that we have some value $v$. It as at this point that we apply
  L\"ob induction to add an induction hypothesis of
  $\later \wpre{\text{\tt push(v)}}{v. Q}$ to our context. We then
  simplify our goal using $\beta$ and bind to
  \[
    \wpre{\text{\tt put v}}{v'. \wpre{\text{\tt match v' with ...}}{Q}}
  \]
  Since we have no specification yet formalized for {\tt push}, we
  proceed to unfold it and begin to work with the internal
  implementation of {\tt put}.
  \[
    \wpre{\text{\tt let off = mk\_offer v in ...}}{v'. \wpre{\text{\tt match v' with ...}}{Q}}
  \]
  We can simplify this again and apply bind
  \[
    \wpre{\text{\tt (v, ref 0)}}
    {v'. \wpre{\text{\tt let off = v' in ...}}{v'. \wpre{\text{\tt match v' with ...}}{Q}}}
  \]
  We can then apply the rule for allocation, so suppose that we have
  some $\ell$ and $\ell \mapsto 0$. We then must show that
  \[
    \wpre{\text{\tt r' := Some (v, l); revoke\_offer (v, l)}}
    {v'. \wpre{\text{\tt match v' with ...}}{Q}}
  \]
  We may then open the invariant
  $\knowInv{\iota'}{\textlog{is\_mailbox}(r', \iota)}$ and apply the
  replace the previous value stored in $r'$ with {\tt (v, l)} and
  reestablish the invariant
  $\knowInv{\iota'}{\textlog{is\_mailbox}(r', \iota)}$. In order to do
  this, we must use $\All vs. P(vs) \vsW Q * P(v :: vs)$ and allocate
  a new invariant $\knowInv{\iota''}{\textlog{stages}_\gamma(l)}$ for
  some $\gamma$ so that we also have $\ownGhost{\gamma}{\exinj(())}$.
  Thus, our goal is now
  \[
    \wpre{\text{\tt revoke\_offer (v, l)}}{v'. \wpre{\text{\tt match v' with ...}}{Q}}
  \]
  If we apply $\beta$ rules we then end up with
  \[
    \wpre{(\text{\tt if cas l 0 2 then Some(v) else None})}{v'. \wpre{\text{\tt match v' with ...}}{Q}}
  \]
  Let us then open up the invariant
  $\knowInv{\iota''}{\textlog{stages}(l)}$. This tells us that there
  \[
    (\ell \mapsto 0 * (\All vs. P(vs) \vsW P (v :: vs) * Q))
    \mathrel{\vee} (\ell \mapsto 1 * Q)
    \mathrel{\vee} \ell \mapsto 1 * \ownGhost{\gamma}{\exinj(())}
    \mathrel{\vee} (\ell \mapsto 2 * \ownGhost{\gamma}{\exinj(())})
  \]
  Let us perform case analysis on these three possible results. We can
  immediately dismiss the last two cases since we have assumed to own
  $\ownGhost{\gamma}{\exinj(())}$ which means that the
  $\ownGhost{\gamma}{\exinj(())}$ coming from this invariant would
  produce a contradiction.

  In the second case, the CAS fails so we are left with $\ell \mapsto
  1$, $\ownGhost{\gamma}{\exinj(())}$, and $Q$. Let us reestablish our
  invariant using the $\ell \mapsto 1$ and
  $\ownGhost{\gamma}{\exinj(())}$ using the second branch. Our goal is then
  \[
    \wpre{(\text{\tt None})}{v'. \wpre{\text{\tt match v' with ...}}{Q}}
  \]
  and simplifying this gives us the obligation
  \[
    \wpre{\text{\tt None}}{Q}
  \]
  but this is trivial to discharge using our assumption of $Q$ that we
  got from the invariant. This case is essentially the position we
  would be in if the side-channel is accepted.


  In the first case, therefore, we must consider what happens if the
  side-channel is not used. Clearly the CAS is successful
  and we can reestablish this invariant by providing
  $\ownGhost{\gamma}{\exinj(())}$ and choosing the last disjunct. In
  this case, we have the assumption $\All vs. P(vs) \vsW P (v :: vs) * Q$
  and the goal
  \[
    \wpre{(\text{\tt Some(v)})}{v'. \wpre{\text{\tt match v' with ...}}{Q}}
  \]
  Therefore, we can apply $\beta$ rules to reduce this to
  \[
    \wpre{\text{\tt let x = !r in let y = Some (v, x) in ...}}{Q}
  \]
  In order to reduce this, we must open the invariant
  $\knowInv{\iota}{\textlog{stack\_inv}(r)}$ which tells us, amongst
  other things, that $r = \ell'$ for some $\ell'$ and that $\ell'
  \mapsto s$ for some value $s$. We can then step our program and
  reestablish the invariant (stepping does not effect the context at
  all) to
  \[
    \wpre{\text{\tt let x = s in let y = Some (v, x) in ...}}{Q}
  \]
  We can now repeatedly simplify our goal to
  \[
    \wpre{\text{\tt if cas r s (Some (v, s)) then () else push v}}{Q}
  \]
  In order to simplify this, we must once again open up the invariant
  $\knowInv{\iota}{\textlog{stack\_inv}(r)}$. This tells us that for
  some $s'$, $L$ and $\ell'$ that
  \[
    r = \ell' * \ell' \mapsto s' * P(L) * \textlog{is\_stack}(L, s')
  \]
  holds. Let us then case on whether or not $s' = s$.

  If it does, then the CAS succeeds, therefore we have
  $\ell' \mapsto \textlog{Some}(v, s)$. Simple logic gives us that
  $\textlog{is\_stack}(L, s')$ implies that
  $\textlog{is\_stack}(v :: L, \textlog{Some}(v, s))$
  holds. In order to reestablish our invariant we must show that $\vs
  \textlog{stack\_inv}(r)$ holds. However, since we have the
  implication $P(L) \vsW P(v :: L)$ it is sufficient to show that
  \[
    r = \ell' * \ell' \mapsto \textlog{Some}(v, s) * P(L)
    * \textlog{is\_stack}(v :: L, \textlog{Some}(v, s))
  \]
  however we have precisely these assumptions so we're done. This
  leaves us with the goal $\wpre{()}{Q}$ which is discharged
  immediately with our assumption of $Q$.


  If these are not equal, then the CAS fails so it's trivial to
  reestablish the invariant. Our goal is then $\wpre{\text{\tt push v}}{Q}$
  but for this we just apply our IH and we're done.
\end{proof}

The Coq formalization of this specification and proof may be found in
{\tt concurrent\_stack4.v}.

%%% Local Variables:
%%% mode: latex
%%% TeX-master: "main"
%%% End:
