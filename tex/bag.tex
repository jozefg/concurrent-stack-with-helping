\section{A First Formalization: A Bag Specification}

The first specification of the concurrent stack really only
specifies the stack's behavior as a \emph{bag}. Nowhere in the
specification is the order of insertion reflected. In a concurrent
setting this is less damning than it might appear because from the
perspective of a single thread it is indeed the case that there is
little connection between the order in which things are inserted and
when they will be removed. This is a direct result of the fact that a
thread must be agnostic to the interference and actions of other
threads operating on the same stack concurrently.

With this in mind, the specification for the stack will be
parameterized by some arbitrary predicate on values,
$P : \Val \to \iProp$. Every element of the stack will satisfy $P$ and
thus our specifications are roughly
\begin{align*}
  P(v) \wand &\wpre{\textlog{push}(v)}[\mathcal{E}]{\TRUE}\\
  &\wpre{\textlog{pop}()\ \ \,}[\mathcal{E}]
  {v. v = \textlog{None} \mathrel{\vee} \Exists v'. v = \textlog{Some}(v') * P(v')}
\end{align*}
Of course the actual specifications must specify {\tt stack}, a
function which returns a tuple a push and pop. This necessitates the
use of a higher order specification, a weakest preconditions whose
post condition contains other weakest preconditions. This is possible
because Iris weakest preconditions are not encoded as a separate sort
of logical proposition but rather as ordinary $\iProp$s. Furthermore,
to make using the specification easier, a common trick with weakest
preconditions is employed. Instead of directly stating something along
the lines of $\wpre{e}[\mathcal{E}]{\Phi}$, instead we introduce a
``cut'',
$\All \Psi. (\All v. \Phi(v) \wand \Psi(v)) \wand \wpre{e}[\mathcal{E}]{\Psi}$. This
makes chaining together multiple weakest preconditions considerably
simpler and avoids gratuitous uses of the rule of consequence. With
all of this said, the first specification for concurrent stacks is
\begin{align*}
  \forall \Phi.\qquad&\\
  (\All f_1 f_2.&
        \wpre{f_1()}{v. v = \textlog{None} \mathrel{\vee} \Exists v'. v = \textlog{Some}(v') * P(v')}\\
  \wand & \All v. P(v) \wand \wpre{f_2(v)}{\TRUE}\\
  \wand & \Phi(f_1, f_2))\\
  \wand \wpre{&\textlog{stack}()}{\Phi}
\end{align*}
Rather than directly verifying this specification, the proof depends
on several helpful lemmas verifying the behavior of offers and
mailboxes. By proving these simple sublemmas the verification of
concurrent stacks can respect the abstraction boundaries constructed by
isolating mailboxes as we have done.

\subsection{Verifying Offers}

The heart of verifying offers is accurately encoding the transition
system described in the previous section. Roughly, an offer can
transition from initial to accepted or from initial to revoked but no
other transitions are to be allowed. Encoding this requires a simple
but interesting application of ghost state. It is possible to encode
an arbitrary state transition system in Iris but in this case a more
specialized approach is simpler.

Specifically, offers will be governed by a proposition
$\textlog{stages}$ which encodes what state of the three an offer is
in. Ghost state is needed to ensure that certain transitions are only
possible for threads with \emph{ownership} of the offer. To do this,
the exclusive monoid on unit will act as a token giving the owner the
right to transition to the original state to the revoked state. The
main thread will have access to $\ownGhost{\gamma}{\exinj(())}$ for
some $\gamma$ and it will be necessary to use this resource to
transition from the initial state to the revoked state. The
proposition encoding the transition system is
\[
  \textlog{stages}_\gamma(v, \ell) \triangleq
  (P(v) * \ell \mapsto 0) \mathrel{\vee}
  \ell \mapsto 1 \mathrel{\vee}
  (\ell \mapsto 2 * \ownGhost{\gamma}{\exinj(())})
\]
Having defined this, the proposition $\textlog{is\_offer}$ is now within
reach.
\[
  \textlog{is\_offer}_\gamma(v) \triangleq
  \Exists v', \ell. v = (v', \ell) *
  \Exists \iota. \knowInv{\iota}{\textlog{stages}_\gamma(v', \ell)}
\]
Notice that $\textlog{is\_offer}$ is clearly persistent, reflecting
the fact that it ought to be shared between multiple threads. This
implies that knowing $\textlog{is\_offer}_\gamma(v)$ does not assert
ownership of any kind, rather, it asserts that for an atomic step of
computation the owner may assume that the offer is in one of the three
states. This sharing provides the motivation for using invariants to
capture $\textlog{stages}$. Without wrapping it in an invariant it
would not be possible to share it between multiple threads. Notice
that both of these propositions are parameterized by a ghost name,
$\gamma$. Each $\gamma$ should uniquely correspond to an offer and
represents the ownership the creator of an offer has over it, namely
the right to revoke it. This is expressed in the specification of {\tt
  mk\_offer}.
\[
  \All v. P(v) \wand
  \wpre{\textlog{mk\_offer}(v)}{v.\ \Exists \gamma.
    \ownGhost{\gamma}{\exinj(())} * \textlog{is\_offer}_\gamma(v)
  }
\]
This reads as that calling {\tt mk\_offer} will allocate an offer
\emph{as well as} returning $\ownGhost{\gamma}{\exinj(())}$ which
represents the right to revoke an offer. This can be seen in the
specification for {\tt revoke\_offer}.
\[
  \All \gamma, v. \textlog{is\_offer}_\gamma(v) * \ownGhost{\gamma}{\exinj(())}
  \wand
  \wpre{\textlog{revoke\_offer}(v)}{v.
    \ v = \textlog{None} \mathrel{\vee}
    \Exists v'. v = \textlog{Some}(v) * P(v')
  }
\]
The specification for {\tt accept\_offer} is remarkably similar except
that it does not require ownership of
$\ownGhost{\gamma}{\exinj(())}$. This is because multiple threads may
call {\tt accept\_offer} even though it will only successfully return
once.
\[
  \All \gamma, v. \textlog{is\_offer}_\gamma(v)
  \wand
  \wpre{\textlog{accept\_offer}(v)}{v.
    \ v = \textlog{None} \mathrel{\vee}
    \Exists v'. v = \textlog{Some}(v) * P(v')
  }
\]
As an illustrative example, we will go through the the derivation of
the specification for $\textlog{mk\_offer}$ and leave the derivations
of the other two specifications as an exercise to the reader.

\begin{thm}
  $\All v. P(v) \wand
  \wpre{\textlog{mk\_offer}(v)}{v.\ \Exists \gamma.
    \ownGhost{\gamma}{\exinj(())} * \textlog{is\_offer}_\gamma(v)
  }$ holds.
\end{thm}
\begin{proof}
  Let us assume that we have some $v \in \Val$ and further that $P(v)$
  holds. We wish to show that
  $\wpre{\textlog{mk\_offer}(v)}{v.\ \Exists \gamma.
    \ownGhost{\gamma}{\exinj(())} * \textlog{is\_offer}_\gamma(v)
  }$ holds. By applying the $\beta$ rule for functions, we must show
  that
  \[
    \wpre{(v, \textlog{ref}(0))}{v.\ \Exists \gamma.
      \ownGhost{\gamma}{\exinj(())} * \textlog{is\_offer}_\gamma(v)
    }
  \]
  We then apply the rule for allocation, so we may assume that we have
  some location $\ell$ and that $\ell \mapsto 0$. We now need to show
  that
  $\wpre{(v, \ell)}{v.\ \Exists \gamma. \ownGhost{\gamma}{\exinj(())} * \textlog{is\_offer}_\gamma(v)}$.
  Now we merely need to show that
  \[
    (v, \ell) \in \Val \mathrel{\wedge}
    \pvs[\mathcal{E}] \Exists \gamma. \ownGhost{\gamma}{\exinj(())} * \textlog{is\_offer}_\gamma((v, \ell))
  \]
  The left-hand side of this conjunction is trivial. For the
  right-hand side, we note that since $\mval(\exinj(()))$ clearly
  holds we have that
  $\pvs[\mathcal{E}] \Exists \gamma. \ownGhost{\gamma}{\exinj(())}$
  holds. Therefore, it suffices to show that
  $\pvs[\mathcal{E}] \textlog{is\_offer}_\gamma((v, \ell))$ holds for
  some $\gamma$. For this, we first note that
  $\textlog{is\_offer}_\gamma((v, \ell))$ is of course equal to
  $\Exists v', \ell'. (v, \ell) = (v', \ell') * \Exists
  \iota. \knowInv{\iota}{\textlog{stages}_\gamma(v', \ell')}$
  So we start by introducing the existential quantifier with $v$ and
  $\ell$ giving us the goal
  \[
    \pvs[\mathcal{E}] (v, \ell) = (v, \ell) * \Exists \iota. \knowInv{\iota}{\textlog{stages}_\gamma(v, \ell)}
  \]
  The equality is obviously true so we merely need to show that
  $\pvs[\mathcal{E}] \Exists \iota. \knowInv{\iota}{\textlog{stages}_\gamma(v, \ell)}$.
  For this, we will use the invariant allocation rule, so we need to show
  that $\later \textlog{stages}_\gamma(v, \ell)$ holds and we're
  done. For this, we prove $\textlog{stages}_\gamma(v, \ell)$ for
  which it suffices to show $P(v) * \ell \mapsto 0$ which we have in our
  assumptions.
\end{proof}

\subsection{Verifying Mailboxes}

Having verified that offers work as intended, the next step is to
verify that the mailbox abstraction built on top of them also
satisfies the intended specification. In order to properly specify
mailboxes, it is necessary to use a similar trick to that of the
specification of stacks. That is, a specification that involves higher
order weakest preconditions and bakes in a cut.
\begin{align}
  \forall \Phi.\qquad& \label{prop:spec1:mailboxspec}\\
  (\All f_1 f_2.&
        (\All v. P(v) \wand \wpre{f_1(v)}{v. v = \textlog{None} \mathrel{\vee} \Exists v'. v = \textlog{Some}(v') * P(v')})
                  \nonumber\\
  \wand & \wpre{f_2()}{v. v = \textlog{None} \mathrel{\vee} \Exists v'. v = \textlog{Some}(v') * P(v')}
          \nonumber\\
  \wand & \Phi(f_1, f_2)) \nonumber\\
  \wand \wpre{&\textlog{mailbox}()}{\Phi} \nonumber
\end{align}
In a small victory for compositional verification, the proof of this
specification is made with no reference to the underlying
implementation of offers, only to the specification previously
proven. Throughout the proof an invariant is maintained governing the
shared mutable cell that contains potential offers. This invariant
enforces that when this cell is full, it contains an offer. It looks
like this
\[
  \textlog{is\_mailbox}(v) \triangleq
  \Exists \ell. v = \ell *
  \ell \mapsto \textlog{None} \mathrel{\vee}
  \Exists v' \gamma. l \mapsto \textlog{Some}(v') * \textlog{is\_offer}_\gamma(v')
\]
This captures the informal notion described above.
\begin{thm}
  Proposition~(\ref{prop:spec1:mailboxspec}) holds.
\end{thm}
\begin{proof}
  For this, we start by applying the $\beta$-rule. This means that in
  addition to our assumption that
  \begin{align*}
  \All f_1 f_2.&
        (\All v. P(v) \wand \wpre{f_1(v)}{v. v = \textlog{None} \mathrel{\vee} \Exists v'. v = \textlog{Some}(v') * P(v')})\\
  \wand & \wpre{f_2()}{v. v = \textlog{None} \mathrel{\vee} \Exists v'. v = \textlog{Some}(v') * P(v')}\\
  \wand & \Phi(f_1, f_2)
  \end{align*}
  our goal is
  \[
    \wpre{\text{\tt let r = ref None in (..., ...)}}{\Phi}
  \]
  We then apply the rule for allocation, so we suppose that we have
  some $\ell$ such that we also have $\ell \mapsto \textlog{None}$.
  Our goal after applying another $\beta$ rule is then of the form
  \[
    \wpre{(..., ...)}{\Phi}
  \]
  We now need to show that $\pvs[\top] \Phi(..., ...)$ holds so we
  take this time to allocate the mailbox invariant as discussed
  above. We will allocate
  $\knowInv{\iota}{\textlog{is\_mailbox}(\ell)}$, for some $\iota$,
  so we must prove that $\later \textlog{is\_mailbox}(\ell)$ holds. We
  first instantiate the existential quantifier with $\ell$ leaving us
  with the goal
  \[
    \ell = \ell * \ell \mapsto \textlog{None} \mathrel{\vee}
    \Exists v', \gamma. l \mapsto \textlog{Some}(v') * \textlog{is\_offer}_\gamma(v')
  \]
  Obviously the equality holds and the left side of the disjunct holds
  by our assumption that $\ell \mapsto \textlog{None}$ so we're done.
  We now apply our original hypothesis leaving us to prove
  \begin{align*}
    \All v. P(v) \wand &\wpre{f_1(v)}{v. v = \textlog{None} \mathrel{\vee} \Exists v'. v = \textlog{Some}(v') * P(v')}\\
    &\wpre{f_2()\ \ }{v. v = \textlog{None} \mathrel{\vee} \Exists v'. v = \textlog{Some}(v') * P(v')}
  \end{align*}
  where we have defined $f_1$ and $f_2$ as

  \noindent\begin{minipage}[t]{0.5\linewidth}
  \begin{lstlisting}
    f1 = rec put v ->
     let off = mk_offer v in
     r := Some off;
     revoke_offer off,
  \end{lstlisting}
  \end{minipage}\hfill
  \begin{minipage}[t]{0.45\linewidth}
  \begin{lstlisting}
    f2 = rec get n ->
     let offopt = !r in
     match offopt with
       None -> None
     | Some x -> accept_offer x
     end
  \end{lstlisting}
  \end{minipage}

  We shall verify the specification for $f_1$ and leave the
  specification of $f_2$ as an exercise. Let us assume that we have
  $P(v)$ for some $v$. We then apply the specification for
  {\tt mk\_offer} to conclude that we must show that
  \begin{align*}
    \textlog{is\_offer}_\gamma&(\textlog{off}) * \ownGhost{\gamma}{\exinj(())} \wand\\
    &\wpre{(\text{\tt r := Some off; revoke\_offer off})}{v. v = \textlog{None} \mathrel{\vee} \Exists v'. v = \textlog{Some}(v') * P(v')}
  \end{align*}
  So let us assume that we know
  $\textlog{is\_offer}_\gamma(\textlog{off})$ as well as
  $\ownGhost{\gamma}{\exinj(())}$. We then open our invariant for the
  single atomic reduction of {\tt r := Some off}. We must then show
  the following
  \begin{align*}
    &\later \textlog{is\_mailbox}(r) \wand\\
    &\ \  \wpre{(\text{\tt r := Some off})}{v. \textlog{is\_mailbox}(r) *
    \wpre{\text{\tt (v; revoke\_offer off)}}
      {v. v = \textlog{None} \mathrel{\vee} \Exists v'. v = \textlog{Some}(v') * P(v')}}
  \end{align*}
  Notice that we have only $\later \textlog{is\_mailbox}(r)$ because
  opening an invariant gives only $\later$ of the stored
  proposition. This will suffice in our case because the rule for
  loading a location does not require $\ell \mapsto v$, only
  $\later \ell \mapsto v$ because $\ell \mapsto v$ is a timeless
  proposition. Weakest preconditions are designed in such a way that
  it is always possible to remove $\later$s from timeless
  propositions.

  Let us then assume that we have $\later \textlog{is\_mailbox}(r)$.
  We know then that there is a location $\ell'$ so that $r = \ell'$
  and $\later \ell \mapsto \textlog{None} \mathrel{\vee}
  \Exists v' \gamma. \later (l \mapsto \textlog{Some}(v') * \textlog{is\_offer}_\gamma(v'))$.
  We then case analyze this disjunction.

  In the first case, we have $\ell' \mapsto \textlog{None}$ so we can
  apply the rule for stores leaving us with the goal
  \begin{align*}
    \ell' \mapsto& \textlog{Some}(\textlog{off}) \wand\\
    &\textlog{is\_mailbox}(\ell') * \wpre{\text{\tt (v; revoke\_offer off)}}
      {v. v = \textlog{None} \mathrel{\vee} \Exists v'. v = \textlog{Some}(v') * P(v')}
  \end{align*}
  First, we prove $\textlog{is\_mailbox}(\ell')$ with our assumptions
  that $\ell' \mapsto \textlog{Some}(\textlog{off})$ and
  $\textlog{is\_offer}_\gamma(\textlog{off})$. This is quite
  straightforward. We prove this existential for $\ell'$. For this we
  must show that
  \[
    \ell = \ell' * \ell \mapsto \textlog{None} \mathrel{\vee}
    \Exists v' \gamma. l \mapsto \textlog{Some}(v') * \textlog{is\_offer}_\gamma(v')
  \]
  but the right disjunct is precisely the assumptions we have. We then
  must show the rest of the goal
  \[
    \wpre{\text{\tt ((); revoke\_offer off)}} {v. v = \textlog{None} \mathrel{\vee} \Exists v'. v = \textlog{Some}(v') * P(v')}
  \]
  For this we apply the $\beta$ and notice that
  $\wpre{\text{\tt revoke\_offer(off)}} {v. v = \textlog{None}
    \mathrel{\vee} \Exists v'. v = \textlog{Some}(v') * P(v')}$
  is precisely the specification we proved earlier for {\tt
    revoke\_offer} and we conveniently have
  $\ownGhost{\gamma}{\exinj(())}$ and $\textlog{is\_offer}(\textlog{off})$
  (remember that it's persistent) so we're done.

  The reasoning for the other disjunct is identical so we elide it
  here.
\end{proof}

\subsection{Verifying Stacks}

We now turn to the verification of stacks themselves. The
specification for these has already been discussed:
\begin{align}
  \forall \Phi.\qquad& \label{prop:spec1:stack}\\
  (\All f_1 f_2.&
        \wpre{f_1()}{v. v = \textlog{None} \mathrel{\vee} \Exists v'. v = \textlog{Some}(v) * P(v)} \nonumber\\
  \wand & \All v. P(v) \wand \wpre{f_2(v)}{\TRUE} \nonumber\\
  \wand & \Phi(f_1, f_2)) \nonumber\\
  \wand \wpre{&\textlog{stack}()}{\Phi} \nonumber
\end{align}
Having verified mailboxes already only a small amount of additional
preparation is needed before actually verifying this
proposition. Specifically, an invariant representing that a memory
cell contains a stack is needed. This is necessary because this cell
will be shared between multiple threads concurrently reading and
writing to it and without an invariant there is no way to reason about
this. The predicate $\textlog{is\_stack}(v)$ used to form the
invariant is defined as follows by guarded recursion
\[
  \textlog{is\_stack}(v) \triangleq
  \MU R. v = \textlog{None} \mathrel{\vee} \Exists h, t. v = \textlog{Some}(h, t) * P(h) * \later R(t)
\]
Having defined this, it is straightforward to define an invariant
enforcing that a location points to a stack.
\[
  \textlog{stack\_inv}(v) \triangleq
  \Exists \ell, v'. v = \ell * \ell \mapsto v' * \textlog{is\_stack}(v')
\]
We turn now to verifying the proposition.
\begin{thm}
  Proposition~(\ref{prop:spec1:stack}) holds.
\end{thm}
\begin{proof}
  For this, we assume first that we have
  \begin{align*}
    \All f_1 f_2.&
        \wpre{f_1()}{v. v = \textlog{None} \mathrel{\vee} \Exists v'. v = \textlog{Some}(v) * P(v)}\\
    \wand & \All v. P(v) \wand \wpre{f_2(v)}{\TRUE}\\
    \wand & \Phi(f_1, f_2))\\
  \end{align*}
  Now we apply the $\beta$ rule to conclude that we must show that
  \[
    \wpre{(\text{\tt let mailbox = mailbox () in E})}{\Phi}
  \]
  We can then apply the bind rule to see that it suffices to prove
  instead
  \[
    \wpre{\mathtt{mailbox()}}{v. \wpre{\text{\tt let mailbox = v in E}}{\Phi}}
  \]
  This is in the form that we can apply our specification for {\tt
    mailbox}. Our goal is now to show that
  \begin{align*}
    \All f_1 f_2.&
    \All v. P(v) \wand \wpre{f_1(v)}{v. v = \textlog{None} \mathrel{\vee} \Exists v'. v = \textlog{Some}(v') * P(v')}\\
    \wand & \wpre{f_2()}{v. v = \textlog{None} \mathrel{\vee} \Exists v'. v = \textlog{Some}(v') * P(v')}\\
    \wand & \wpre{\text{\tt let mailbox = (f1, f2) in E}}{\Phi}
  \end{align*}
  This is the advantage provided by specifying our lemmas with a cut
  built in. The mailbox lemma is immediately applicable without
  further manipulation. Let us assume that we have such an $f_1$ and
  $f_2$ and furthermore that the above two specifications holds for
  $f_1$ and $f_2$. We now attempt to prove that
  $\wpre{\text{\tt let mailbox = (f1, f2) in E}}{\Phi}$.  We next
  apply the $\beta$ rule for {\tt let} to transform our goal into
  \[
    \wpre{\text{\tt let get = f1 in let put = f2 in let r = ref None in (P1, P2)}}{\Phi}
  \]
  Now again we can apply our $\beta$ rules to conclude that we need to
  show that the following holds
  \[
    \wpre{\text{\tt let r = ref None in ([f1/get]P1, [f2/put]P2)}}{\Phi}
  \]
  We now apply the allocation rule, so suppose that we have some
  $\ell$, we must then show that if $\ell \mapsto \textlog{None}$ then
  $\wpre{\text{\tt ([f1/get]P1, [f2/put]P2)}}{\Phi}$ holds. Before
  attempting to prove this though, we allocate then invariant
  $\textlog{stack\_inv}(\ell)$. This is easy to do because allocating
  this invariant requires showing that
  \[
    \later \Exists \ell', v'. \ell = \ell' * \ell' \mapsto v' * \textlog{is\_stack}(v')
  \]
  Let us prove this by proving that
  $\Exists \ell' v'. \ell = \ell' * \ell' \mapsto v' *
  \textlog{is\_stack}(v')$ directly and instantiating the existential
  with $\ell$ and $\textlog{None}$. Our remaining obligation is just
  to show that
  \[
    \MU R. v = \textlog{None} \mathrel{\vee} \Exists h, t. v = \textlog{Some}(h, t) * P(h) * \later R(t)
  \]
  But the left disjunct is just reflexivity. We may then assume
  $\knowInv{\iota}{\textlog{is\_stack}(r)}$. Now our goal is simply
  of the form
  \[
    \wpre{\text{\tt ([f1/get]P1, [f2/put]P2)}}{\Phi}
  \]
  We, however, have an assumption that
  \begin{align*}
    \All f_1 f_2.&
    \wpre{f_1()}{v. v = \textlog{None} \mathrel{\vee} \Exists v'. v = \textlog{Some}(v') * P(v')}\\
    \wand & \All v. P(v) \wand \wpre{f_2(v)}{\TRUE}\\
    \wand & \Phi(f_1, f_2))
  \end{align*}
  Now we apply this to our current goal leaving us to show that
  \[
    \wpre{\text{\tt [f1/get]P1()}}{v. v = \textlog{None} \mathrel{\vee} \Exists v'. v = \textlog{Some}(v') * P(v')}
    \qquad
    \All v. P(v) \wand \wpre{\text{\tt [f1/get]P2(v)}}{\TRUE}
  \]
  We will prove both of these separately now.
  \begin{enumerate}
  \item $\wpre{\text{\tt [f1/get]P1()}}{v. v = \textlog{None} \mathrel{\vee} \Exists v'. v = \textlog{Some}(v') * P(v')}$\\
    First, let us apply L\"ob induction to get the assumption
    \[
      \later \wpre{\text{\tt [f1/get]P1()}}{v. v = \textlog{None} \mathrel{\vee} \Exists v'. v = \textlog{Some}(v') * P(v')}
    \]
    We will make use of this assumption in the case where we are
    forced to loop due to contention on the main stack. We next apply
    the bind rule to change our goal to
    \[
      \wpre{\text{\tt f1 ()}}
      {v. \wpre{(\text{\tt match v with ...})}{v. v = \textlog{None} \mathrel{\vee} \Exists v'. v = \textlog{Some}(v') * P(v')}}
    \]
    We can now apply the assumption that we have for $f_1$, namely we
    now must prove the entailment.
    \[
      \All v. (v = \textlog{None} \mathrel{\vee} \Exists v'. v = \textlog{Some}(v') * P(v')) \wand
      \wpre{(\text{\tt match v with ...})}{v. v = \textlog{None} \mathrel{\vee} \Exists v'. v = \textlog{Some}(v') * P(v')}
    \]
    Let us assume that we have some $v$ and that
    $v = \textlog{None} \mathrel{\vee} \Exists v'. v = \textlog{Some}(v') * P(v')$ holds.
    We now case on this disjunction.

    Let us first consider the case where
    $\Exists v'. v = \textlog{Some}(v') * P(v')$ holds. We now need to
    show can then destruct this existential telling us that there is
    some $v'$ so that $v = \textlog{Some}(v')$ and $P(v')$
    holds. Rewriting by $v = \textlog{Some}(v')$ we can then apply the
    $\beta$ rule for matches to transform our goal into
    \[
      \wpre{(\text{\tt Some(v')})}{v. v = \textlog{None} \mathrel{\vee} \Exists v'. v = \textlog{Some}(v') * P(v')}
    \]
    However since this is a value it suffices to prove that
    \[
      \Exists v''. \textlog{Some}(v'') = \textlog{Some}(v') * P(v'')
    \]
    however our assumptions give this immediately.

    Now consider the case where $v = \textlog{None}$. In this case we
    can rewrite again by this equality and apply the $\beta$ rule for
    match to conclude our goal is
    \[
      \wpre{(\text{\tt
          match !r with
            None -> ...
          | Some hd -> ... end})}
      {v. v = \textlog{None} \mathrel{\vee} \Exists v'. v = \textlog{Some}(v') * P(v')}
    \]
    We now again apply our binding rule to rewrite our conclusion to
    the form
    \[
      \wpre{(\text{\tt !r})}{v.
       \wpre{\text{\tt match v with ...}}
       {v. v = \textlog{None} \mathrel{\vee} \Exists v'. v = \textlog{Some}(v') * P(v')}
      }
    \]
    Now that we have an atomic operation, we can apply our invariant
    rule to open the invariant that we have about $r$. Let us assume
    that we have $\later \textlog{stack\_inv}(r)$. We can then rewrite
    this to
    \[
      \Exists \ell, v'. \later (v = \ell * \ell \mapsto v' * \textlog{is\_stack}(v'))
    \]
    which is of course equivalent to
    \[
      \Exists \ell, v'. \later v = \ell * \later \ell \mapsto v' * \later \textlog{is\_stack}(v')
    \]
    For this, we then conclude that there exists some $v'$ and $\ell$
    so that these conditions hold. We then can step our goal to
    \[
      \wpre{(\text{\tt v'})}{v.
       \wpre{\text{\tt match v with ...}}
       {v. v = \textlog{None} \mathrel{\vee} \Exists v'. v = \textlog{Some}(v') * P(v')}
      }
    \]
    and remove the $\later$s from our assumptions so that we have
    $v = \ell * \later \ell \mapsto v' * \textlog{is\_stack}(v')$.
    Next we unfold $\textlog{is\_stack}(v')$ to conclude that either
    $v' = \textlog{None}$ or there is a $h, t$ so that
    $v = \textlog{Some}(h, t)$ where $P(h)$ and
    $\later \textlog{is\_stack}(t)$ holds.

    In the first case, we rewrite our goal to
    \[
      \wpre{(\text{\tt None})}{v.
       \wpre{\text{\tt match v with ...}}
       {v. v = \textlog{None} \mathrel{\vee} \Exists v'. v = \textlog{Some}(v') * P(v')}
      }
    \]
    Finally, we can easily reestablish our invariant as required since
    we have not consumed any resources, that is, we can prove
    $\later \textlog{stack\_inv}(r)$ using our assumptions that
    $v' = \textlog{None}$ with $r = \ell$ and $\ell \mapsto v'$.
    Finally we now apply the $\beta$ rule for match transforming our
    goal into
    \[
      \wpre{\mathtt{None}}{v. v = \textlog{None} \mathrel{\vee} \Exists v'. v = \textlog{Some}(v') * P(v')}
    \]
    which is immediately established because
    $\textlog{None} = \textlog{None}$ obviously holds.

    Let us consider the second case. We again rewrite by this equality
    and reestablish our invariant. We can reestablish our invariant
    because, again, we have just destructed it without consuming any
    of the resources it provided. Our goal is then
    \[
      \wpre{\text{\tt match Some(h, t) with ...}}
      {v. v = \textlog{None} \mathrel{\vee} \Exists v'. v = \textlog{Some}(v') * P(v')}
    \]
    Notice that because we packed up all of our knowledge of $h, t$
    back into our invariant, we have no information currently recorded
    about $h$ or $t$. Next we apply the $\beta$ rule for match as well
    as a few simple $\beta$ reductions for projections to get
    \[
      \wpre{\text{\tt if cas r (Some v') t then Some h else pop n}}
      {v. v = \textlog{None} \mathrel{\vee} \Exists v'. v = \textlog{Some}(v') * P(v')}
    \]
    We then again apply our bind rule to change our goal to
    \[
      \wpre{\text{\tt cas r (Some v') t}}{
        v. \wpre{(\text{\tt ...})}
        {v. v = \textlog{None} \mathrel{\vee} \Exists v'. v = \textlog{Some}(v') * P(v')}
      }
    \]
    We again open our invariant
    $\knowInv{\iota}{\textlog{stack\_inv}(r)}$. This gives us that
    $\later \textlog{stack\_inv}(r)$. Unfolding all of this, we get
    that there is some $\ell'$ and some $v''$
    \[
      r = \ell' * \later \ell' \mapsto v'' * \later \textlog{is\_stack}(v'')
    \]
    Let us then case on whether or not $v' = v''$.

    In the first case we can successfully compute our CAS so our goal
    is
    \[
      \wpre{(\text{\tt if true then ... else ...})}
      {v. v = \textlog{None} \mathrel{\vee} \Exists v'. v = \textlog{Some}(v') * P(v')}
    \]
    where now $r = \ell'$ and $\ell' \mapsto t$. Since we have taken a
    step, we may strip the $\later$s off of our assumption. We now
    turn to reestablishing our invariant which is nontrivial since we
    have changed what $r$ points to. We must show that
    \[
      \later \Exists \ell, v. r = \ell * \ell \mapsto v * \later \textlog{is\_stack}(v)
    \]
    For this, we instantiate it with $\ell = \ell'$ and $v = t$. We
    then have the first two goals by assumption so we merely need to
    show that $\later \textlog{is\_stack}(t)$ holds. For this, we
    unfold our assumption that $\textlog{is\_stack}(v'')$ holds. Since
    $v'' = \textlog{Some}(h, t)$ we must have that
    \[
      \Exists h', t'. v'' = (h', t') * P(h') * \later \textlog{is\_stack}(t')
    \]
    Unfolding this, we use injectivity to conclude that $h' = h$ and
    $t' = t$ so we have $P(h') * \later \textlog{is\_stack}(t')$. The
    latter gives us immediately what we need to reestablish the
    invariant. We also hold on to the assumption that $P(h)$ holds and
    return to our goal that
    \[
      \wpre{(\text{\tt if true then Some(h) else pop(n)})}
      {v. v = \textlog{None} \mathrel{\vee} \Exists v'. v = \textlog{Some}(v') * P(v')}
    \]
    so we step this to conclude that we must show
    \[
      \textlog{Some}(h) = \textlog{None} \mathrel{\vee} \Exists v'. \textlog{Some}(h) = \textlog{Some}(v') * P(v')
    \]
    however the right disjunct of this holds by assumption so we're
    done with this case.

    Finally, we consider the case that $v' \neq v''$. In this case our
    CAS fails so we can step our goal to
    \[
      \wpre{(\text{\tt if false then Some(h) else pop(n)})}
      {v. v = \textlog{None} \mathrel{\vee} \Exists v'. v = \textlog{Some}(v') * P(v')}
    \]
    it is trivial to reestablish our invariant since we know that
    $\textlog{stack\_inv}(v'')$ still holds as we have not consumed
    any of the resources. Finally we apply the $\beta$ rule for if to
    turn out goal into
    \[
      \wpre{(\text{\tt pop(n)})}{v. v = \textlog{None} \mathrel{\vee} \Exists v'. v = \textlog{Some}(v') * P(v')}
    \]
    but now we apply our IH that we created earlier using L\"ob
    induction and we're done.
  \item $\All v. P(v) \wand \wpre{\text{\tt [f1/get]P2(v)}}{\TRUE}$\\
    This is left as an exercise to the reader as it is strictly
    simpler than the above proof.
  \end{enumerate}
\end{proof}

The Coq formalization of all of this can be found in {\tt
  concurrent\_stack2.v}.

%%% Local Variables:
%%% mode: latex
%%% TeX-master: "main"
%%% End:
