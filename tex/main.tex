\documentclass[12pt]{amsart}
\usepackage{iris}
\usepackage{tikz}
\usetikzlibrary{arrows}
\usepackage{chngcntr}
\usepackage{fullpage}
\usepackage{amsmath, amssymb, amsthm}
\usepackage[foot]{amsaddr}
\usepackage{listings, xcolor}
\usepackage[T1]{fontenc}
\usepackage{natbib}

\counterwithout{figure}{section}

\lstset{
    language=Caml,
    basicstyle=\ttfamily,
    keywordstyle=\color{blue}\ttfamily,
    stringstyle=\color{red}\ttfamily,
    commentstyle=\color{darkgreen}\ttfamily,
    breaklines=true,
    columns=flexible,
    literate={->}{$\to$\ }1,
    xleftmargin=\leftmargini,
  }

\newtheorem{thm}{Theorem}[section]
\newtheorem{cor}[thm]{Corollary}
\newtheorem{lem}[thm]{Lemma}
\newtheorem{remark}[thm]{Remark}
\newtheorem{defn}[thm]{Definition}

\newcommand{\disjoint}{\mathrel{\bot}}

\title{Formalizing Concurrent Stacks with Helping: \\
  A Case Study in Iris}
\author{Daniel Gratzer}
\address{Carnegie Mellon University}
\author{Mathias H{\o}ier \and Ale{\u s} Bizjak \and Lars Birkedal}
\address{Aarhus University}
\date{\today}

\begin{document}
\begin{abstract}
  Iris is an expressive higher-order separation logic designed for the
  verification of concurrent, imperative programs. As a demonstration
  of this, we present a formalization of a common concurrent
  data-structure: a stack. The implementation is lock-free and uses
  helping to avoid contention. All of the examples are formalized in
  Coq and demonstrate how Iris can be used to give expressive,
  higher-order specifications of advanced concurrent data-structures.
\end{abstract}
\maketitle

\section{Introduction}

Iris is a general, base logic suitable for constructing a variety of
program logics. It has been used extensively to verify properties of
various programs, in particular those making use of fine-grained
concurrency. In this case study we will verify a larger example of
such a concurrent program which uses atomic operations to implement a
lock-free data-structure.

Despite how it is normally used, the Iris logic itself is designed
quite independently of any particular programming language or indeed
the notion of a program logic at all. It is a standard higher-order
logic supplemented with several modalities. An introduction to it may
be found in~\citet{Jung:2017} or~\citet{Jung:2015}. In this paper we
will assume familiarity with Iris as well as one of the main languages
it has been used to study: a concurrent ML supplemented with general,
mutable reference and a CAS operation. We will focus on showing how to
use the logic provided by Iris to examine a real, substantive
concurrent data-structure: a concurrent stack with helping. We hope
that this case study will give a flavor for how verification of such
data-structures normally proceeds in Iris. In order to do this we will
give several different specifications for the stack. Each
specification will be more precise and correspondingly complex than
the previous one. Inevitably, the specifications will force us to
choose between the modularity of the proof and the precision with
which our specification describes the data-structure. The first
specification, which ignores the stack ordering of the data-structure
and merely treats it as a bag, allows for a simple and direct
verification that follows the abstractions set up in the code
itself. The final specification which captures far more of the
behavior of the stack forces us to ignore such abstractions.

To begin with, we will start by reviewing the code and informally
arguing towards its correctness. We will then specify it as a bag and
provide a fully worked out proof that the code satisfies the
specification. We will then prove a more precise specification making
use of atomic updates to an abstract stack for our data structure with
and without helping. Finally, we will conclude with the same precise
specification applied to the full data-structure supplemented with
helping.

%%% Local Variables:
%%% mode: latex
%%% TeX-master: "main"
%%% End:

\section{A Concurrent Stack with Helping}

The abstract data type that we're implementing is that of a
stack. Therefore, it will have two operations, {\tt push} and
{\tt pop}. The main complication of our data structure is that {\tt
  push} and {\tt pop} must be thread-safe.

In order to handle this in the implementation, we will make use of the
fact that we have an atomic compare-and-swap operation on
reference cells in our language. By storing our stack in such a cell,
we can retrieve it, modify it as necessary, and attempt to atomically
replace the old stack with our new stack. As is usual with
fine-grained concurrency, it is entirely possible that halfway
through our operation another thread might modify the stack. In this
case, the attempt to replace the old stack with the new stack using a
CAS instruction will fail. The typical way to handle this is to loop
and attempt the operation from the beginning, though in high
contention situations this can be a major delay and cause long running
operations to live-lock.

In order to partially mitigate this, it would be ideal if some of our
concurrent operations could avoid dealing with the main cell holding
the stack entirely. For example, if one thread is attempting a push at
the same time another is attempting a pop, they will fight each other
for ownership of the cell even though it would be perfectly valid for
one thread to simply hand the other the value. In a high contention
situation with many threads pushing and popping at once, this can be
quite a common problem. We handle this by introducing a
\emph{side-channel} for threads to communicate along. That is, before
a thread attempts to work with the main stack it will, for instance,
check whether or not someone is offering a value along the
side-channel that it could just take instead. Similarly a thread which
is pushing a value onto the stack will offer its value on the
side-channel temporarily in an attempt to avoid having to compete for
the main stack. This scheme reduces contention on the main atomic cell
and thus improves performance.

\subsection{Mailboxes for Offers}
In order to do this, before designing a stack we first implement a
small API for side-channels. A side-channel has the following
operations
\begin{enumerate}
\item An offer can be \emph{created} with an initial value.
\item An offer can be accepted, marking the offer as taken and
  returning the underlying value.
\item Once created, an offer can be revoked which will prevent anyone
  from accepting the offer and return the underlying value to the
  thread.
\end{enumerate}
Of course, all of these operations have to be thread-safe! That is, it
must be safe for an offer to be attempted to be accepted by multiple
threads at once, an offer needs to be able to be revoked while it's
being accepted, and so on. We choose to represent an offer as a tuple
of the actual value the offer contains and a reference to an int. The
underlying integer may take one of 3 values, either $0$, $1$ or
$2$. Therefore, an offer of the form $(v, \ell)$ with $\ell \mapsto 0$
is the initial state of an offer, no one has attempted to take it nor
has it been revoked. Someone may attempt to take the offer in which
case they will use a CAS to switch $\ell$ from $0$ to $1$, leading to
the accepted state of an offer which is $(v, \ell)$ so that
$\ell \mapsto 1$.  Revoking is almost identical but instead of
switching from $0$ to $1$ instead we switch to $2$. Since both
revoking and accepting an offer demand the offer to be in the initial
state it is impossible anything other than exactly one accept or one
revoke to succeed. The actual code for this is in
Figure~\ref{fig:code:offer} and the transition graph sketched above is
illustrated in Figure~\ref{fig:code:offertransgraph}.
\begin{figure}
\begin{lstlisting}
  mk_offer = fun v -> (v, ref 0)
  revoke_offer =
    fun v ->
      if cas (snd v) 0 2
      then Some (fst v)
      else None
  accept_offer =
    fun v ->
      if cas (snd v) 0 1
      then Some (fst v)
      else None
\end{lstlisting}
\caption{The implementation of offers used to construct side-channels.}
\label{fig:code:offer}
\end{figure}
\begin{figure}
  \begin{tikzpicture}[->,>=stealth',shorten >=1pt,auto,node distance=4cm,
    thick,main node/.style={circle,draw,font=\sffamily\Large\bfseries}]

    \node[main node] (1) at (0, 0)    {(-, 0)};
    \node[main node] (2) at (4,  1.0) {(-, 1)};
    \node[main node] (3) at (4, -1.0) {(-, 2)};
    \path[every node/.style={font=\sffamily\small}]
    (1) edge[bend left]  node[pos=0.8, text depth=0.4cm,  left] {accept} (2)
        edge[bend right] node[pos=0.8, text depth=-0.4cm, left] {revoke} (3);
  \end{tikzpicture}
  \caption{The states an offer may be in.}
  \label{fig:code:offertransgraph}
\end{figure}
The pattern of offering something, immediately revoking it, and
returning the value if the revoke was successful is sufficiently
common that we can encapsulate it in an abstraction called a
\emph{mailbox}. The idea is that a mailbox is built around an
underlying cell containing an offer and that it provides two functions
which, respectively, briefly put a new offer out and check for such an
offer. The code for this may be found in
Figure~\ref{fig:code:mailbox}. One small technical detail is that we
have designed this as a constructor which returns two closures which
manipulate the same reference cell. This simplifies the process of
using a mailbox for stacks where we only have one mailbox at a
time. It is not, however, fundamentally different than an
implementation more in the style of the offers above.
\begin{figure}
\begin{lstlisting}
  mailbox = fun () ->
    let r = ref None in
    (rec put v ->
       let off = mk_offer v in
       r := Some off;
       revoke_offer off,
     rec get n ->
       let offopt = !r in
       match offopt with
         None -> None
       | Some x -> accept_offer x
       end)
\end{lstlisting}
\caption{The implementation of mailboxes which provide a convenient
  wrapper over offers.}
\label{fig:code:mailbox}
\end{figure}
\subsection{The Implementation of the Stack}
With an implementation of offers, it is easy to code up the concurrent
stack. The idea for implementing a side-channel is to have a
designated cell for threads to put pending offers in. This way, when a
thread comes along to push a value onto the stack it will first create
an offer using the above API and put it into the given cell. It will
then immediately revoke it to see if another thread has accepted the
offer in the meantime and if none has, it will proceed with the
pushing algorithm. The dual process suffices for a thread seeking to
pop. The code for the stack is in Figure~\ref{fig:code:stack}. Notice
that this too is written in a similar style to that of mailboxes, a
make function which returns two closures for the operations rather
than having them separately accept a stack as an argument.
\begin{figure}
\begin{lstlisting}
  stack = fun () ->
    let mailbox = mailbox () in
    let put = fst mailbox in
    let get = snd mailbox in
    let r = ref None in
    (rec pop n ->
       match get () with
         None ->
         (match !r with
            None -> None
          | Some hd =>
            if cas r (Some hd) (snd hd)
            then Some (fst hd)
            else pop n
          end)
       | Some x -> Some x
       end,
      rec push n ->
        match put n with
          None -> ()
        | Some n ->
          let r' = !r in
          let r'' = Some (n, r') in
          if cas r r' r''
          then ()
          else push n
        end)
\end{lstlisting}
\caption{The implementation of the concurrent stack.}
\label{fig:code:stack}
\end{figure}

%%% Local Variables:
%%% mode: latex
%%% TeX-master: "main"
%%% End:

\section{A First Formalization: A Bag Specification}

The first specification of the concurrent stack really only
specifies the stack's behavior as a \emph{bag}. Nowhere in the
specification is the order of insertion reflected. In a concurrent
setting this is less damning than it might appear because from the
perspective of a single thread it is indeed the case that there is
little connection between the order in which things are inserted and
when they will be removed. This is a direct result of the fact that a
thread must be agnostic to the interference and actions of other
threads operating on the same stack concurrently.

With this in mind, the specification for the stack will be
parameterized by some arbitrary predicate on values,
$P : \Val \to \iProp$. Every element of the stack will satisfy $P$ and
thus our specifications are roughly
\begin{align*}
  P(v) \wand &\wpre{\textlog{push}(v)}[\mathcal{E}]{\TRUE}\\
  &\wpre{\textlog{pop}()\ \ \,}[\mathcal{E}]
  {v. v = \textlog{None} \mathrel{\vee} \Exists v'. v = \textlog{Some}(v') * P(v')}
\end{align*}
Of course the actual specifications must specify {\tt stack}, a
function which returns a tuple a push and pop. This necessitates the
use of a higher order specification, a weakest preconditions whose
post condition contains other weakest preconditions. This is possible
because Iris weakest preconditions are not encoded as a separate sort
of logical proposition but rather as ordinary $\iProp$s. Furthermore,
to make using the specification easier, a common trick with weakest
preconditions is employed. Instead of directly stating something along
the lines of $\wpre{e}[\mathcal{E}]{\Phi}$, instead we introduce a
``cut'',
$\All \Psi. (\All v. \Phi(v) \wand \Psi(v)) \wand \wpre{e}[\mathcal{E}]{\Psi}$. This
makes chaining together multiple weakest preconditions considerably
simpler and avoids gratuitous uses of the rule of consequence. With
all of this said, the first specification for concurrent stacks is
\begin{align*}
  \forall \Phi.\qquad&\\
  (\All f_1 f_2.&
        \wpre{f_1()}{v. v = \textlog{None} \mathrel{\vee} \Exists v'. v = \textlog{Some}(v') * P(v')}\\
  \wand & \All v. P(v) \wand \wpre{f_2(v)}{\TRUE}\\
  \wand & \Phi(f_1, f_2))\\
  \wand \wpre{&\textlog{stack}()}{\Phi}
\end{align*}
Rather than directly verifying this specification, the proof depends
on several helpful lemmas verifying the behavior of offers and
mailboxes. By proving these simple sublemmas the verification of
concurrent stacks can respect the abstraction boundaries constructed by
isolating mailboxes as we have done.

\subsection{Verifying Offers}

The heart of verifying offers is accurately encoding the transition
system described in the previous section. Roughly, an offer can
transition from initial to accepted or from initial to revoked but no
other transitions are to be allowed. Encoding this requires a simple
but interesting application of ghost state. It is possible to encode
an arbitrary state transition system in Iris but in this case a more
specialized approach is simpler.

Specifically, offers will be governed by a proposition
$\textlog{stages}$ which encodes what state of the three an offer is
in. Ghost state is needed to ensure that certain transitions are only
possible for threads with \emph{ownership} of the offer. To do this,
the exclusive monoid on unit will act as a token giving the owner the
right to transition to the original state to the revoked state. The
main thread will have access to $\ownGhost{\gamma}{\exinj(())}$ for
some $\gamma$ and it will be necessary to use this resource to
transition from the initial state to the revoked state. The
proposition encoding the transition system is
\[
  \textlog{stages}_\gamma(v, \ell) \triangleq
  (P(v) * \ell \mapsto 0) \mathrel{\vee}
  \ell \mapsto 1 \mathrel{\vee}
  (\ell \mapsto 2 * \ownGhost{\gamma}{\exinj(())})
\]
Having defined this, the proposition $\textlog{is\_offer}$ is now within
reach.
\[
  \textlog{is\_offer}_\gamma(v) \triangleq
  \Exists v', \ell. v = (v', \ell) *
  \Exists \iota. \knowInv{\iota}{\textlog{stages}_\gamma(v', \ell)}
\]
Notice that $\textlog{is\_offer}$ is clearly persistent, reflecting
the fact that it ought to be shared between multiple threads. This
implies that knowing $\textlog{is\_offer}_\gamma(v)$ does not assert
ownership of any kind, rather, it asserts that for an atomic step of
computation the owner may assume that the offer is in one of the three
states. This sharing provides the motivation for using invariants to
capture $\textlog{stages}$. Without wrapping it in an invariant it
would not be possible to share it between multiple threads. Notice
that both of these propositions are parameterized by a ghost name,
$\gamma$. Each $\gamma$ should uniquely correspond to an offer and
represents the ownership the creator of an offer has over it, namely
the right to revoke it. This is expressed in the specification of {\tt
  mk\_offer}.
\[
  \All v. P(v) \wand
  \wpre{\textlog{mk\_offer}(v)}{v.\ \Exists \gamma.
    \ownGhost{\gamma}{\exinj(())} * \textlog{is\_offer}_\gamma(v)
  }
\]
This reads as that calling {\tt mk\_offer} will allocate an offer
\emph{as well as} returning $\ownGhost{\gamma}{\exinj(())}$ which
represents the right to revoke an offer. This can be seen in the
specification for {\tt revoke\_offer}.
\[
  \All \gamma, v. \textlog{is\_offer}_\gamma(v) * \ownGhost{\gamma}{\exinj(())}
  \wand
  \wpre{\textlog{revoke\_offer}(v)}{v.
    \ v = \textlog{None} \mathrel{\vee}
    \Exists v'. v = \textlog{Some}(v) * P(v')
  }
\]
The specification for {\tt accept\_offer} is remarkably similar except
that it does not require ownership of
$\ownGhost{\gamma}{\exinj(())}$. This is because multiple threads may
call {\tt accept\_offer} even though it will only successfully return
once.
\[
  \All \gamma, v. \textlog{is\_offer}_\gamma(v)
  \wand
  \wpre{\textlog{accept\_offer}(v)}{v.
    \ v = \textlog{None} \mathrel{\vee}
    \Exists v'. v = \textlog{Some}(v) * P(v')
  }
\]
As an illustrative example, we will go through the the derivation of
the specification for $\textlog{mk\_offer}$ and leave the derivations
of the other two specifications as an exercise to the reader.

\begin{thm}
  $\All v. P(v) \wand
  \wpre{\textlog{mk\_offer}(v)}{v.\ \Exists \gamma.
    \ownGhost{\gamma}{\exinj(())} * \textlog{is\_offer}_\gamma(v)
  }$ holds.
\end{thm}
\begin{proof}
  Let us assume that we have some $v \in \Val$ and further that $P(v)$
  holds. We wish to show that
  $\wpre{\textlog{mk\_offer}(v)}{v.\ \Exists \gamma.
    \ownGhost{\gamma}{\exinj(())} * \textlog{is\_offer}_\gamma(v)
  }$ holds. By applying the $\beta$ rule for functions, we must show
  that
  \[
    \wpre{(v, \textlog{ref}(0))}{v.\ \Exists \gamma.
      \ownGhost{\gamma}{\exinj(())} * \textlog{is\_offer}_\gamma(v)
    }
  \]
  We then apply the rule for allocation, so we may assume that we have
  some location $\ell$ and that $\ell \mapsto 0$. We now need to show
  that
  $\wpre{(v, \ell)}{v.\ \Exists \gamma. \ownGhost{\gamma}{\exinj(())} * \textlog{is\_offer}_\gamma(v)}$.
  Now we merely need to show that
  \[
    (v, \ell) \in \Val \mathrel{\wedge}
    \pvs[\mathcal{E}] \Exists \gamma. \ownGhost{\gamma}{\exinj(())} * \textlog{is\_offer}_\gamma((v, \ell))
  \]
  The left-hand side of this conjunction is trivial. For the
  right-hand side, we note that since $\mval(\exinj(()))$ clearly
  holds we have that
  $\pvs[\mathcal{E}] \Exists \gamma. \ownGhost{\gamma}{\exinj(())}$
  holds. Therefore, it suffices to show that
  $\pvs[\mathcal{E}] \textlog{is\_offer}_\gamma((v, \ell))$ holds for
  some $\gamma$. For this, we first note that
  $\textlog{is\_offer}_\gamma((v, \ell))$ is of course equal to
  $\Exists v', \ell'. (v, \ell) = (v', \ell') * \Exists
  \iota. \knowInv{\iota}{\textlog{stages}_\gamma(v', \ell')}$
  So we start by introducing the existential quantifier with $v$ and
  $\ell$ giving us the goal
  \[
    \pvs[\mathcal{E}] (v, \ell) = (v, \ell) * \Exists \iota. \knowInv{\iota}{\textlog{stages}_\gamma(v, \ell)}
  \]
  The equality is obviously true so we merely need to show that
  $\pvs[\mathcal{E}] \Exists \iota. \knowInv{\iota}{\textlog{stages}_\gamma(v, \ell)}$.
  For this, we will use the invariant allocation rule, so we need to show
  that $\later \textlog{stages}_\gamma(v, \ell)$ holds and we're
  done. For this, we prove $\textlog{stages}_\gamma(v, \ell)$ for
  which it suffices to show $P(v) * \ell \mapsto 0$ which we have in our
  assumptions.
\end{proof}

\subsection{Verifying Mailboxes}

Having verified that offers work as intended, the next step is to
verify that the mailbox abstraction built on top of them also
satisfies the intended specification. In order to properly specify
mailboxes, it is necessary to use a similar trick to that of the
specification of stacks. That is, a specification that involves higher
order weakest preconditions and bakes in a cut.
\begin{align}
  \forall \Phi.\qquad& \label{prop:spec1:mailboxspec}\\
  (\All f_1 f_2.&
        (\All v. P(v) \wand \wpre{f_1(v)}{v. v = \textlog{None} \mathrel{\vee} \Exists v'. v = \textlog{Some}(v') * P(v')})
                  \nonumber\\
  \wand & \wpre{f_2()}{v. v = \textlog{None} \mathrel{\vee} \Exists v'. v = \textlog{Some}(v') * P(v')}
          \nonumber\\
  \wand & \Phi(f_1, f_2)) \nonumber\\
  \wand \wpre{&\textlog{mailbox}()}{\Phi} \nonumber
\end{align}
In a small victory for compositional verification, the proof of this
specification is made with no reference to the underlying
implementation of offers, only to the specification previously
proven. Throughout the proof an invariant is maintained governing the
shared mutable cell that contains potential offers. This invariant
enforces that when this cell is full, it contains an offer. It looks
like this
\[
  \textlog{is\_mailbox}(v) \triangleq
  \Exists \ell. v = \ell *
  \ell \mapsto \textlog{None} \mathrel{\vee}
  \Exists v' \gamma. l \mapsto \textlog{Some}(v') * \textlog{is\_offer}_\gamma(v')
\]
This captures the informal notion described above.
\begin{thm}
  Proposition~(\ref{prop:spec1:mailboxspec}) holds.
\end{thm}
\begin{proof}
  For this, we start by applying the $\beta$-rule. This means that in
  addition to our assumption that
  \begin{align*}
  \All f_1 f_2.&
        (\All v. P(v) \wand \wpre{f_1(v)}{v. v = \textlog{None} \mathrel{\vee} \Exists v'. v = \textlog{Some}(v') * P(v')})\\
  \wand & \wpre{f_2()}{v. v = \textlog{None} \mathrel{\vee} \Exists v'. v = \textlog{Some}(v') * P(v')}\\
  \wand & \Phi(f_1, f_2)
  \end{align*}
  our goal is
  \[
    \wpre{\text{\tt let r = ref None in (..., ...)}}{\Phi}
  \]
  We then apply the rule for allocation, so we suppose that we have
  some $\ell$ such that we also have $\ell \mapsto \textlog{None}$.
  Our goal after applying another $\beta$ rule is then of the form
  \[
    \wpre{(..., ...)}{\Phi}
  \]
  We now need to show that $\pvs[\top] \Phi(..., ...)$ holds so we
  take this time to allocate the mailbox invariant as discussed
  above. We will allocate
  $\knowInv{\iota}{\textlog{is\_mailbox}(\ell)}$, for some $\iota$,
  so we must prove that $\later \textlog{is\_mailbox}(\ell)$ holds. We
  first instantiate the existential quantifier with $\ell$ leaving us
  with the goal
  \[
    \ell = \ell * \ell \mapsto \textlog{None} \mathrel{\vee}
    \Exists v', \gamma. l \mapsto \textlog{Some}(v') * \textlog{is\_offer}_\gamma(v')
  \]
  Obviously the equality holds and the left side of the disjunct holds
  by our assumption that $\ell \mapsto \textlog{None}$ so we're done.
  We now apply our original hypothesis leaving us to prove
  \begin{align*}
    \All v. P(v) \wand &\wpre{f_1(v)}{v. v = \textlog{None} \mathrel{\vee} \Exists v'. v = \textlog{Some}(v') * P(v')}\\
    &\wpre{f_2()\ \ }{v. v = \textlog{None} \mathrel{\vee} \Exists v'. v = \textlog{Some}(v') * P(v')}
  \end{align*}
  where we have defined $f_1$ and $f_2$ as

  \noindent\begin{minipage}[t]{0.5\linewidth}
  \begin{lstlisting}
    f1 = rec put v ->
     let off = mk_offer v in
     r := Some off;
     revoke_offer off,
  \end{lstlisting}
  \end{minipage}\hfill
  \begin{minipage}[t]{0.45\linewidth}
  \begin{lstlisting}
    f2 = rec get n ->
     let offopt = !r in
     match offopt with
       None -> None
     | Some x -> accept_offer x
     end
  \end{lstlisting}
  \end{minipage}

  We shall verify the specification for $f_1$ and leave the
  specification of $f_2$ as an exercise. Let us assume that we have
  $P(v)$ for some $v$. We then apply the specification for
  {\tt mk\_offer} to conclude that we must show that
  \begin{align*}
    \textlog{is\_offer}_\gamma&(\textlog{off}) * \ownGhost{\gamma}{\exinj(())} \wand\\
    &\wpre{(\text{\tt r := Some off; revoke\_offer off})}{v. v = \textlog{None} \mathrel{\vee} \Exists v'. v = \textlog{Some}(v') * P(v')}
  \end{align*}
  So let us assume that we know
  $\textlog{is\_offer}_\gamma(\textlog{off})$ as well as
  $\ownGhost{\gamma}{\exinj(())}$. We then open our invariant for the
  single atomic reduction of {\tt r := Some off}. We must then show
  the following
  \begin{align*}
    &\later \textlog{is\_mailbox}(r) \wand\\
    &\ \  \wpre{(\text{\tt r := Some off})}{v. \textlog{is\_mailbox}(r) *
    \wpre{\text{\tt (v; revoke\_offer off)}}
      {v. v = \textlog{None} \mathrel{\vee} \Exists v'. v = \textlog{Some}(v') * P(v')}}
  \end{align*}
  Notice that we have only $\later \textlog{is\_mailbox}(r)$ because
  opening an invariant gives only $\later$ of the stored
  proposition. This will suffice in our case because the rule for
  loading a location does not require $\ell \mapsto v$, only
  $\later \ell \mapsto v$ because $\ell \mapsto v$ is a timeless
  proposition. Weakest preconditions are designed in such a way that
  it is always possible to remove $\later$s from timeless
  propositions.

  Let us then assume that we have $\later \textlog{is\_mailbox}(r)$.
  We know then that there is a location $\ell'$ so that $r = \ell'$
  and $\later \ell \mapsto \textlog{None} \mathrel{\vee}
  \Exists v' \gamma. \later (l \mapsto \textlog{Some}(v') * \textlog{is\_offer}_\gamma(v'))$.
  We then case analyze this disjunction.

  In the first case, we have $\ell' \mapsto \textlog{None}$ so we can
  apply the rule for stores leaving us with the goal
  \begin{align*}
    \ell' \mapsto& \textlog{Some}(\textlog{off}) \wand\\
    &\textlog{is\_mailbox}(\ell') * \wpre{\text{\tt (v; revoke\_offer off)}}
      {v. v = \textlog{None} \mathrel{\vee} \Exists v'. v = \textlog{Some}(v') * P(v')}
  \end{align*}
  First, we prove $\textlog{is\_mailbox}(\ell')$ with our assumptions
  that $\ell' \mapsto \textlog{Some}(\textlog{off})$ and
  $\textlog{is\_offer}_\gamma(\textlog{off})$. This is quite
  straightforward. We prove this existential for $\ell'$. For this we
  must show that
  \[
    \ell = \ell' * \ell \mapsto \textlog{None} \mathrel{\vee}
    \Exists v' \gamma. l \mapsto \textlog{Some}(v') * \textlog{is\_offer}_\gamma(v')
  \]
  but the right disjunct is precisely the assumptions we have. We then
  must show the rest of the goal
  \[
    \wpre{\text{\tt ((); revoke\_offer off)}} {v. v = \textlog{None} \mathrel{\vee} \Exists v'. v = \textlog{Some}(v') * P(v')}
  \]
  For this we apply the $\beta$ and notice that
  $\wpre{\text{\tt revoke\_offer(off)}} {v. v = \textlog{None}
    \mathrel{\vee} \Exists v'. v = \textlog{Some}(v') * P(v')}$
  is precisely the specification we proved earlier for {\tt
    revoke\_offer} and we conveniently have
  $\ownGhost{\gamma}{\exinj(())}$ and $\textlog{is\_offer}(\textlog{off})$
  (remember that it's persistent) so we're done.

  The reasoning for the other disjunct is identical so we elide it
  here.
\end{proof}

\subsection{Verifying Stacks}

We now turn to the verification of stacks themselves. The
specification for these has already been discussed:
\begin{align}
  \forall \Phi.\qquad& \label{prop:spec1:stack}\\
  (\All f_1 f_2.&
        \wpre{f_1()}{v. v = \textlog{None} \mathrel{\vee} \Exists v'. v = \textlog{Some}(v) * P(v)} \nonumber\\
  \wand & \All v. P(v) \wand \wpre{f_2(v)}{\TRUE} \nonumber\\
  \wand & \Phi(f_1, f_2)) \nonumber\\
  \wand \wpre{&\textlog{stack}()}{\Phi} \nonumber
\end{align}
Having verified mailboxes already only a small amount of additional
preparation is needed before actually verifying this
proposition. Specifically, an invariant representing that a memory
cell contains a stack is needed. This is necessary because this cell
will be shared between multiple threads concurrently reading and
writing to it and without an invariant there is no way to reason about
this. The predicate $\textlog{is\_stack}(v)$ used to form the
invariant is defined as follows by guarded recursion
\[
  \textlog{is\_stack}(v) \triangleq
  \MU R. v = \textlog{None} \mathrel{\vee} \Exists h, t. v = \textlog{Some}(h, t) * P(h) * \later R(t)
\]
Having defined this, it is straightforward to define an invariant
enforcing that a location points to a stack.
\[
  \textlog{stack\_inv}(v) \triangleq
  \Exists \ell, v'. v = \ell * \ell \mapsto v' * \textlog{is\_stack}(v')
\]
We turn now to verifying the proposition.
\begin{thm}
  Proposition~(\ref{prop:spec1:stack}) holds.
\end{thm}
\begin{proof}
  For this, we assume first that we have
  \begin{align*}
    \All f_1 f_2.&
        \wpre{f_1()}{v. v = \textlog{None} \mathrel{\vee} \Exists v'. v = \textlog{Some}(v) * P(v)}\\
    \wand & \All v. P(v) \wand \wpre{f_2(v)}{\TRUE}\\
    \wand & \Phi(f_1, f_2))\\
  \end{align*}
  Now we apply the $\beta$ rule to conclude that we must show that
  \[
    \wpre{(\text{\tt let mailbox = mailbox () in E})}{\Phi}
  \]
  We can then apply the bind rule to see that it suffices to prove
  instead
  \[
    \wpre{\mathtt{mailbox()}}{v. \wpre{\text{\tt let mailbox = v in E}}{\Phi}}
  \]
  This is in the form that we can apply our specification for {\tt
    mailbox}. Our goal is now to show that
  \begin{align*}
    \All f_1 f_2.&
    \All v. P(v) \wand \wpre{f_1(v)}{v. v = \textlog{None} \mathrel{\vee} \Exists v'. v = \textlog{Some}(v') * P(v')}\\
    \wand & \wpre{f_2()}{v. v = \textlog{None} \mathrel{\vee} \Exists v'. v = \textlog{Some}(v') * P(v')}\\
    \wand & \wpre{\text{\tt let mailbox = (f1, f2) in E}}{\Phi}
  \end{align*}
  This is the advantage provided by specifying our lemmas with a cut
  built in. The mailbox lemma is immediately applicable without
  further manipulation. Let us assume that we have such an $f_1$ and
  $f_2$ and furthermore that the above two specifications holds for
  $f_1$ and $f_2$. We now attempt to prove that
  $\wpre{\text{\tt let mailbox = (f1, f2) in E}}{\Phi}$.  We next
  apply the $\beta$ rule for {\tt let} to transform our goal into
  \[
    \wpre{\text{\tt let get = f1 in let put = f2 in let r = ref None in (P1, P2)}}{\Phi}
  \]
  Now again we can apply our $\beta$ rules to conclude that we need to
  show that the following holds
  \[
    \wpre{\text{\tt let r = ref None in ([f1/get]P1, [f2/put]P2)}}{\Phi}
  \]
  We now apply the allocation rule, so suppose that we have some
  $\ell$, we must then show that if $\ell \mapsto \textlog{None}$ then
  $\wpre{\text{\tt ([f1/get]P1, [f2/put]P2)}}{\Phi}$ holds. Before
  attempting to prove this though, we allocate then invariant
  $\textlog{stack\_inv}(\ell)$. This is easy to do because allocating
  this invariant requires showing that
  \[
    \later \Exists \ell', v'. \ell = \ell' * \ell' \mapsto v' * \textlog{is\_stack}(v')
  \]
  Let us prove this by proving that
  $\Exists \ell' v'. \ell = \ell' * \ell' \mapsto v' *
  \textlog{is\_stack}(v')$ directly and instantiating the existential
  with $\ell$ and $\textlog{None}$. Our remaining obligation is just
  to show that
  \[
    \MU R. v = \textlog{None} \mathrel{\vee} \Exists h, t. v = \textlog{Some}(h, t) * P(h) * \later R(t)
  \]
  But the left disjunct is just reflexivity. We may then assume
  $\knowInv{\iota}{\textlog{is\_stack}(r)}$. Now our goal is simply
  of the form
  \[
    \wpre{\text{\tt ([f1/get]P1, [f2/put]P2)}}{\Phi}
  \]
  We, however, have an assumption that
  \begin{align*}
    \All f_1 f_2.&
    \wpre{f_1()}{v. v = \textlog{None} \mathrel{\vee} \Exists v'. v = \textlog{Some}(v') * P(v')}\\
    \wand & \All v. P(v) \wand \wpre{f_2(v)}{\TRUE}\\
    \wand & \Phi(f_1, f_2))
  \end{align*}
  Now we apply this to our current goal leaving us to show that
  \[
    \wpre{\text{\tt [f1/get]P1()}}{v. v = \textlog{None} \mathrel{\vee} \Exists v'. v = \textlog{Some}(v') * P(v')}
    \qquad
    \All v. P(v) \wand \wpre{\text{\tt [f1/get]P2(v)}}{\TRUE}
  \]
  We will prove both of these separately now.
  \begin{enumerate}
  \item $\wpre{\text{\tt [f1/get]P1()}}{v. v = \textlog{None} \mathrel{\vee} \Exists v'. v = \textlog{Some}(v') * P(v')}$\\
    First, let us apply L\"ob induction to get the assumption
    \[
      \later \wpre{\text{\tt [f1/get]P1()}}{v. v = \textlog{None} \mathrel{\vee} \Exists v'. v = \textlog{Some}(v') * P(v')}
    \]
    We will make use of this assumption in the case where we are
    forced to loop due to contention on the main stack. We next apply
    the bind rule to change our goal to
    \[
      \wpre{\text{\tt f1 ()}}
      {v. \wpre{(\text{\tt match v with ...})}{v. v = \textlog{None} \mathrel{\vee} \Exists v'. v = \textlog{Some}(v') * P(v')}}
    \]
    We can now apply the assumption that we have for $f_1$, namely we
    now must prove the entailment.
    \[
      \All v. (v = \textlog{None} \mathrel{\vee} \Exists v'. v = \textlog{Some}(v') * P(v')) \wand
      \wpre{(\text{\tt match v with ...})}{v. v = \textlog{None} \mathrel{\vee} \Exists v'. v = \textlog{Some}(v') * P(v')}
    \]
    Let us assume that we have some $v$ and that
    $v = \textlog{None} \mathrel{\vee} \Exists v'. v = \textlog{Some}(v') * P(v')$ holds.
    We now case on this disjunction.

    Let us first consider the case where
    $\Exists v'. v = \textlog{Some}(v') * P(v')$ holds. We now need to
    show can then destruct this existential telling us that there is
    some $v'$ so that $v = \textlog{Some}(v')$ and $P(v')$
    holds. Rewriting by $v = \textlog{Some}(v')$ we can then apply the
    $\beta$ rule for matches to transform our goal into
    \[
      \wpre{(\text{\tt Some(v')})}{v. v = \textlog{None} \mathrel{\vee} \Exists v'. v = \textlog{Some}(v') * P(v')}
    \]
    However since this is a value it suffices to prove that
    \[
      \Exists v''. \textlog{Some}(v'') = \textlog{Some}(v') * P(v'')
    \]
    however our assumptions give this immediately.

    Now consider the case where $v = \textlog{None}$. In this case we
    can rewrite again by this equality and apply the $\beta$ rule for
    match to conclude our goal is
    \[
      \wpre{(\text{\tt
          match !r with
            None -> ...
          | Some hd -> ... end})}
      {v. v = \textlog{None} \mathrel{\vee} \Exists v'. v = \textlog{Some}(v') * P(v')}
    \]
    We now again apply our binding rule to rewrite our conclusion to
    the form
    \[
      \wpre{(\text{\tt !r})}{v.
       \wpre{\text{\tt match v with ...}}
       {v. v = \textlog{None} \mathrel{\vee} \Exists v'. v = \textlog{Some}(v') * P(v')}
      }
    \]
    Now that we have an atomic operation, we can apply our invariant
    rule to open the invariant that we have about $r$. Let us assume
    that we have $\later \textlog{stack\_inv}(r)$. We can then rewrite
    this to
    \[
      \Exists \ell, v'. \later (v = \ell * \ell \mapsto v' * \textlog{is\_stack}(v'))
    \]
    which is of course equivalent to
    \[
      \Exists \ell, v'. \later v = \ell * \later \ell \mapsto v' * \later \textlog{is\_stack}(v')
    \]
    For this, we then conclude that there exists some $v'$ and $\ell$
    so that these conditions hold. We then can step our goal to
    \[
      \wpre{(\text{\tt v'})}{v.
       \wpre{\text{\tt match v with ...}}
       {v. v = \textlog{None} \mathrel{\vee} \Exists v'. v = \textlog{Some}(v') * P(v')}
      }
    \]
    and remove the $\later$s from our assumptions so that we have
    $v = \ell * \later \ell \mapsto v' * \textlog{is\_stack}(v')$.
    Next we unfold $\textlog{is\_stack}(v')$ to conclude that either
    $v' = \textlog{None}$ or there is a $h, t$ so that
    $v = \textlog{Some}(h, t)$ where $P(h)$ and
    $\later \textlog{is\_stack}(t)$ holds.

    In the first case, we rewrite our goal to
    \[
      \wpre{(\text{\tt None})}{v.
       \wpre{\text{\tt match v with ...}}
       {v. v = \textlog{None} \mathrel{\vee} \Exists v'. v = \textlog{Some}(v') * P(v')}
      }
    \]
    Finally, we can easily reestablish our invariant as required since
    we have not consumed any resources, that is, we can prove
    $\later \textlog{stack\_inv}(r)$ using our assumptions that
    $v' = \textlog{None}$ with $r = \ell$ and $\ell \mapsto v'$.
    Finally we now apply the $\beta$ rule for match transforming our
    goal into
    \[
      \wpre{\mathtt{None}}{v. v = \textlog{None} \mathrel{\vee} \Exists v'. v = \textlog{Some}(v') * P(v')}
    \]
    which is immediately established because
    $\textlog{None} = \textlog{None}$ obviously holds.

    Let us consider the second case. We again rewrite by this equality
    and reestablish our invariant. We can reestablish our invariant
    because, again, we have just destructed it without consuming any
    of the resources it provided. Our goal is then
    \[
      \wpre{\text{\tt match Some(h, t) with ...}}
      {v. v = \textlog{None} \mathrel{\vee} \Exists v'. v = \textlog{Some}(v') * P(v')}
    \]
    Notice that because we packed up all of our knowledge of $h, t$
    back into our invariant, we have no information currently recorded
    about $h$ or $t$. Next we apply the $\beta$ rule for match as well
    as a few simple $\beta$ reductions for projections to get
    \[
      \wpre{\text{\tt if cas r (Some v') t then Some h else pop n}}
      {v. v = \textlog{None} \mathrel{\vee} \Exists v'. v = \textlog{Some}(v') * P(v')}
    \]
    We then again apply our bind rule to change our goal to
    \[
      \wpre{\text{\tt cas r (Some v') t}}{
        v. \wpre{(\text{\tt ...})}
        {v. v = \textlog{None} \mathrel{\vee} \Exists v'. v = \textlog{Some}(v') * P(v')}
      }
    \]
    We again open our invariant
    $\knowInv{\iota}{\textlog{stack\_inv}(r)}$. This gives us that
    $\later \textlog{stack\_inv}(r)$. Unfolding all of this, we get
    that there is some $\ell'$ and some $v''$
    \[
      r = \ell' * \later \ell' \mapsto v'' * \later \textlog{is\_stack}(v'')
    \]
    Let us then case on whether or not $v' = v''$.

    In the first case we can successfully compute our CAS so our goal
    is
    \[
      \wpre{(\text{\tt if true then ... else ...})}
      {v. v = \textlog{None} \mathrel{\vee} \Exists v'. v = \textlog{Some}(v') * P(v')}
    \]
    where now $r = \ell'$ and $\ell' \mapsto t$. Since we have taken a
    step, we may strip the $\later$s off of our assumption. We now
    turn to reestablishing our invariant which is nontrivial since we
    have changed what $r$ points to. We must show that
    \[
      \later \Exists \ell, v. r = \ell * \ell \mapsto v * \later \textlog{is\_stack}(v)
    \]
    For this, we instantiate it with $\ell = \ell'$ and $v = t$. We
    then have the first two goals by assumption so we merely need to
    show that $\later \textlog{is\_stack}(t)$ holds. For this, we
    unfold our assumption that $\textlog{is\_stack}(v'')$ holds. Since
    $v'' = \textlog{Some}(h, t)$ we must have that
    \[
      \Exists h', t'. v'' = (h', t') * P(h') * \later \textlog{is\_stack}(t')
    \]
    Unfolding this, we use injectivity to conclude that $h' = h$ and
    $t' = t$ so we have $P(h') * \later \textlog{is\_stack}(t')$. The
    latter gives us immediately what we need to reestablish the
    invariant. We also hold on to the assumption that $P(h)$ holds and
    return to our goal that
    \[
      \wpre{(\text{\tt if true then Some(h) else pop(n)})}
      {v. v = \textlog{None} \mathrel{\vee} \Exists v'. v = \textlog{Some}(v') * P(v')}
    \]
    so we step this to conclude that we must show
    \[
      \textlog{Some}(h) = \textlog{None} \mathrel{\vee} \Exists v'. \textlog{Some}(h) = \textlog{Some}(v') * P(v')
    \]
    however the right disjunct of this holds by assumption so we're
    done with this case.

    Finally, we consider the case that $v' \neq v''$. In this case our
    CAS fails so we can step our goal to
    \[
      \wpre{(\text{\tt if false then Some(h) else pop(n)})}
      {v. v = \textlog{None} \mathrel{\vee} \Exists v'. v = \textlog{Some}(v') * P(v')}
    \]
    it is trivial to reestablish our invariant since we know that
    $\textlog{stack\_inv}(v'')$ still holds as we have not consumed
    any of the resources. Finally we apply the $\beta$ rule for if to
    turn out goal into
    \[
      \wpre{(\text{\tt pop(n)})}{v. v = \textlog{None} \mathrel{\vee} \Exists v'. v = \textlog{Some}(v') * P(v')}
    \]
    but now we apply our IH that we created earlier using L\"ob
    induction and we're done.
  \item $\All v. P(v) \wand \wpre{\text{\tt [f1/get]P2(v)}}{\TRUE}$\\
    This is left as an exercise to the reader as it is strictly
    simpler than the above proof.
  \end{enumerate}
\end{proof}

The Coq formalization of all of this can be found in {\tt
  concurrent\_stack2.v}.

%%% Local Variables:
%%% mode: latex
%%% TeX-master: "main"
%%% End:

\section{A Second Formalization: View-Shifts without Helping}

The specification proven above has a major defect, it doesn't express
any sort of ordering on the underlying collection of
objects. Informally, it is quite clear that something popped out of
the stack must have been pushed in at some point. This can be argued
by a parametricity-type argument about $P$. Nothing stops us, however,
from ascribing this specification to a concurrent queue or even
something that permutes its elements randomly!

In a concurrent setting this may seem like the best possible
specification though. After all, suppose we have the following code.
\begin{lstlisting}
   push(1);
   push(2);
   let x = pop () in
   let y = pop () in
   (x, y)
\end{lstlisting}
It is perfectly possible for this code to evaluate to either $(1, 2)$
or $(2, 1)$ or even $(6, 7)$. This is because another thread could
come along between the time when the {\tt push}s have been executed,
perform an arbitrary combination of {\tt push}s and {\tt pop}s on the
stack before the original thread can execute its two {\tt pop}s. There
is a far more serious flaw in the specification though: it is
satisfied by operations which always return {\tt None} and discard the
input!

What is needed is a specification that reflects the fact that there is
an underlying stack that may be accessed atomically and the behavior
of the function is determined by the state of this stack at some
particular point. This is related to the standard concurrency idea of
\emph{linearization} where a complicated operation can be reduced to a
single atomic interaction. This means that several complex,
overlapping concurrent operations on a data structure can be logically
linearly ordered.

To begin experimenting with this new form of specification, we will
start by verifying a concurrent stack without helping. This simplifies
the verification considerably. Since, furthermore, helping is an
invisible optimization any good specification for a stack with helping
is an equally good specification for a stack without helping.

In order to do define the specification we shift from parameterizing
our specifications from properties of elements to properties of the
abstract stack that our data structure represents. This abstract stack
can be represented with a simple list at the level of the logic. It
will then be assumed that Iris is supplemented with lists of values, a
basic elimination operation on them (called $\textlog{foldr}$), and
the basic rules of equality for it.

Turning now to the question of what specifications ought to be
concretely, they will have to make use of \emph{view-shifts}. This is
a feature of the Iris base logic which generalizes simple implication,
$\wand$, to allow for the updating of ghost state and the usage of
invariants. These view-shifts are useful in defining the specification
because these precisely capture the idea of a linearization point. If
we supply an operation with a viewshift
$\All L. P(L) \vsW[\mathcal{E}] P(L) * Q$ and that operation is
equipped with the specification
\[
  \knowInv{\iota}{P(L)} \wand \wpre{...}{Q}
\]
then at some point in the code, for an atomic operation the invariant
$\iota$ will be opened and the view-shift will be used. This must
happen because it is the only way to produce a $Q$ to complete this
specification. This view-shift then isolates the reasoning that will be
done during that atomic step using the proposition held by the
invariant. Furthermore, affinity means that such reasoning can only be
applied once. This interaction with the proposition isolated by the
invariant is precisely a linearization point; it's the only time which
we manipulate the shared state guarded by the invariant. These
specifications provide a much more flexible way of interacting with
concurrent functions because they, in effect, capture any
specification which contains the same critical element of
\begin{enumerate}
\item A single abstract property of the stack represented by the
  data-structure.
\item The post conditions are implied (with the possible manipulation
  of ghost state) by the state of abstract data structure at the
  moment it is accessed.
\end{enumerate}

The specification for a push operation might look like this
\[
  \All v.
  (\All vs. P(vs) \vsW[\top \setminus \iota] Q * P(v :: vs)) \wand
  \wpre{\text{\tt push(v)}}{v. Q}
\]
The $\setminus \iota$ can be safely ignored for now. In other words,
if given a means of atomically switching from $P(vs)$ to
$Q * P(v :: vs)$ we can produce a $Q$ in our post condition. If we
want to specialize this to what we had earlier, then $Q = \TRUE$ and
$P(L) = \All x \in L. P(x)$ gives us the specification we used to have
for {\tt push}. The real advantage of this style of specification is
that we can express a lot more than this though. For an example of how
this might be done, see~\citet{Svendsen:2013} which uses these
specification to automatically derive precise \emph{sequential}
specifications from the concurrent ones without undue effort! It is of
course a balance to create an expressive and yet general specification
but the fact that this specification can encapsulate both the
sequential case as well as the highly concurrent bag-like
specification is evidence for its utility.

Turning now to the question of specifying the invariants and
predicates we need to make these specification work, we will need a
new definition of $\textlog{is\_stack}$ and
$\textlog{stack\_inv}$. The idea is that the stack invariant will
contain that there is some stack in a mutable cell representing an
abstract stack $vs$ so that $P(vs)$ holds. That is,
\begin{align*}
  \textlog{is\_stack}([], v) &\triangleq v = \textlog{None}\\
  \textlog{is\_stack}(x :: L, v) &\triangleq v =
  \Exists t. v = \textlog{Some}(x, t) * \textlog{is\_stack}(L, t)\\
  \textlog{stack\_inv}(v) &\triangleq
  \Exists \ell, v', L. v = \ell * \ell \mapsto v' * P(L) * \textlog{is\_stack}(L, v')
\end{align*}
Here we have parameterized our construction by the property we
maintain about the abstract stack, $P$. Let us further assume that we
have some arbitrary $Q$, $Q_1$, and $Q_2$ which will parameterize our
specification. They are specified up front simply to avoid the tedium
of writing quantifiers for them over and over again but no
restrictions need to be imposed on them. By varying $P$, along with
$Q$, $Q_1$ and $Q_2$ it is possible to recover our original
specification as a concurrent bag amongst others.

Therefore, the specification implies that we take the $P(L)$ contained
in the invariant and atomically update it to $P(x :: L)$ before
rebundling the whole thing back into the $\textlog{stack\_inv}$. The
full specification for the stack data structure again makes use of the
``cut'' trick seen earlier.
\begin{align}
  \forall \Phi.\qquad& \label{prop:vsnohelp:spec}\\
  (\All f_1 f_2.& \nonumber\\
        (&(\All vs. P(v :: vs) \vsW[\top \setminus \iota] Q_1(v) * P(vs)) \wand \nonumber\\
         &(\All vs. P([]) \vsW[\top \setminus \iota] Q_2 * P([])) \wand\nonumber\\
         &\wpre{f_1()}{v. v = \textlog{None} * Q_2 \mathrel{\vee} \Exists v'. v = \textlog{Some}(v') * Q_1(v')}\nonumber\\
  \wand \quad (&\All v. (\All vs. P(vs) \vsW[\top \setminus \iota] Q * P(v :: vs)) \wand \wpre{f_2(v)}{v. Q}) \nonumber\\
  \wand \quad \phantom{(}& \Phi(f_1, f_2))\nonumber\\
  \wand P([&]) \nonumber\\
  \wand \wpre{&\textlog{stack}()}{\Phi}\nonumber
\end{align}
The code that will be verified is slightly modified from the previous
version. It is shown in Figure~\ref{fig:vsnohelp:code}
\begin{figure}
  \begin{lstlisting}
  stack = fun () ->
    let r = ref None in
    (rec pop n ->
       match !r with
          None -> None
        | Some hd =>
          if cas r (Some hd) (snd hd)
          then Some (fst hd)
          else pop n
        end,
      rec push n ->
        let r' = !r in
        let r'' = Some (n, r') in
        if cas r r' r''
        then ()
        else push n)
  \end{lstlisting}
  \caption{Concurrent stack \emph{without} helping.}
  \label{fig:vsnohelp:code}
\end{figure}

\begin{thm}
  Proposition~(\ref{prop:vsnohelp:spec}) holds.
\end{thm}
\begin{proof}
  As before, we start by stepping our program, leaving us with the
  goal
  \[
    \wpre{\text{\tt let r = ref None in ...}}{\Phi}
  \]
  with the assumption
  \begin{align*}
    \All f_1 f_2.& \\
    (&(\All vs. P(v :: vs) \vsW[\top \setminus \iota] Q_1(v) * P(vs)) \wand \\
     &(\All vs. P([]) \vsW[\top \setminus \iota] Q_2 * P([])) \wand\\
     &\wpre{f_1()}{v. v = \textlog{None} * Q_1 \mathrel{\vee} \Exists v'. v = \textlog{Some}(v') * Q_1(v')}\\
    \wand \quad (&\All v. (\All vs. P(vs) \vsW[\top \setminus \iota] Q * P(v :: vs)) \wand \wpre{f_2(v)}{v. Q}) \\
    \wand \quad \phantom{(}& \Phi(f_1, f_2)
  \end{align*}
  as well as $P([])$. We then apply the application rule giving us
  some $\ell$ so that $\ell \mapsto \textlog{None}$. We take this time
  to establish the stack invariant for $r$, that is, for some $\iota$
  that $\knowInv{\iota}{\textlog{stack\_inv}(r)}$. In order to do this
  we must show that
  \[
    \later (\Exists \ell', v, L. \ell = \ell' * \ell' \mapsto v * P(L) * \textlog{is\_stack}(L, v))
  \]
  holds according to the invariant allocation rule.

  We will prove this for $\ell$, $\textlog{None}$, and $[]$
  respectively. We then merely must show that the following holds.
  \[
    P([]) * \textlog{is\_stack}([], \textlog{None})
  \]
  However the left-side of this conjunction is an assumption and the
  right-hand side is simply true by reflexivity. We then, returning to
  our main goal, need to show that
  \[
    \wpre{\text{\tt (..., ...)}}{\Phi}
  \]
  holds, and for this we apply our assumption. This leaves us with two
  goals, showing that our specification holds for push and pop. We
  will consider only the case for {\tt push} as an illustrative
  example. That is, we want to show for an arbitrary that $v$
  \begin{align*}
    (\All L. &P(L) \vsW[\top \setminus \iota] P(v :: L) * Q) \wand\\
    &\wpre{(\text{\tt (rec push n -> let r' = !r in let r'' = Some (n, r') in ...) v})}{Q}
  \end{align*}
  We now take a moment to use L\"ob induction so we may assume
  \begin{align*}
    \later (\All v, L. &P(L) \vsW[\top \setminus \iota] P(v :: L) * Q) \wand\\
    &\wpre{(\text{\tt (rec push n -> let r' = !r in let r'' = Some (n, r') in ...) v})}{Q}
  \end{align*}
  Now, we can shift this to
  \[
    \wpre{\text{\tt !r}}{v'.
      \wpre{(\text{\tt let r' = v' in let r'' = Some (v, r') in ...})}{Q}
    }
  \]
  At this point we open the invariant
  $\knowInv{\iota}{\textlog{stack\_inv}(r)}$. This gives us
  that there is some $\ell$, $v''$ and $L$ so that
  \[
    \later (r = \ell * \ell \mapsto v'' * P(L) * \textlog{is\_stack}(L, v''))
  \]
  Using this, we can step our goal to
  \[
    \wpre{(\text{\tt let r' = v'' in let r'' = Some (v, r') in ...})}{Q}
  \]
  and reestablish our invariant trivially since we have not consumed
  any resources. In this position we can step our goal to
  \[
    \wpre{(\text{\tt if cas r v'' (Some (v, v'')) then () else push v})}{Q}
  \]
  At this point we open up our invariant again, giving us
  \[
    \later (r = \ell * \ell \mapsto v''' * P(L) * \textlog{is\_stack}(L, v'''))
  \]
  for some $v'''$. We then case on whether or not $v'' = v'''$.

  First consider the case that this does hold. In this case, we can
  step our CAS successfully giving us the new assumption that
  $\ell \mapsto \textlog{Some}(v, v'')$ and the obligation to
  reestablish our invariant and show that $Q$ holds. For this we first
  apply
  \[
    P(L) \vsW[\top \setminus \iota] P(v :: L) * Q
  \]
  to our assumption that $P(L)$ holds (we may strip of the later since
  we have taken a step of computation). This gives us that
  $\pvs[\top \setminus \iota] P(v :: L) * Q$ holds. We then have that
  $\textlog{is\_stack}(v :: L, \textlog{Some}(v, v''))$ holds because
  we have assumed that $\textlog{is\_stack}(L, v'')$ holds with
  $v'' = v'''$. This, combined with our assumption that
  $\ell \mapsto \textlog{Some}(v, v'')$ gives us that
  $\pvs[\top \setminus \iota] \textlog{stack\_inv}(r)$ holds so we
  may can reestablish our invariant with using $Q$ which we then use
  to discharge the remaining postcondition.

  If instead $v'' \neq v'''$ then we can reduce our goal to
  \[
    \wpre{(\text{\tt if false then () else push v})}{Q}
  \]
  and immediately reestablish our invariant because we have not
  changed anything. But then stepping and applying our induction
  hypothesis immediately gives us the desired conclusion.
\end{proof}

The Coq formalization of this specification and proof may be found in
{\tt concurrent\_stack3.v}.

%%% Local Variables:
%%% mode: latex
%%% TeX-master: "main"
%%% End:

\section{A Third Formalization: View-Shifts with Helping}

It is a straightforward exercise to adapt the previous proof to work
with helping using the same setup as the previous proof. Indeed, since
helping is an invisible optimization to a concurrent data structure
the same specification ought to apply. The only question is how to
modify the specifications we give to mailboxes in order to suitably
handle these view-shifts.

The primary difference is in the definition of the invariants for
offers and mailboxes. Rather than having them pass around ownership of
$P(v)$ for some $P$ where $v$ is the value they contain, they have to
pass around the right to perform certain view-shifts. This
unfortunately means that the proofs for mailboxes and offers become
entangled with that of stacks to a much larger degree. This is why the
Coq development does not isolate them into separate lemmas any
more.

The new invariant for offers is now defined using
\[
  \textlog{stages}_\gamma(\ell, v) \triangleq
  (\ell \mapsto 0 * (\All vs. P(vs) \vsW P (v :: vs) * Q))
  \mathrel{\vee} (\ell \mapsto 1 * Q)
  \mathrel{\vee} (\ell \mapsto 1 * \ownGhost{\gamma}{\exinj(())})
  \mathrel{\vee} (\ell \mapsto 2 * \ownGhost{\gamma}{\exinj(())})
\]
This means that either one of two interactions is possible
\begin{enumerate}
\item A thread without ownership of $\ownGhost{\gamma}{\exinj(())}$
  attempts to switch from $\ell \mapsto 0$ to $\ell \mapsto 1$. In doing so,
  it is obligated to take the view-shift
  $\All vs. P(vs) \vsW (v :: vs) * Q)$ and place the resulting $Q$
  back in the invariant.
\item The thread which owns $\ownGhost{\gamma}{\exinj(())}$ may
  attempt to revoke this offer, switching it from $\ell \mapsto 0$ to
  $\ell \mapsto 2$. If it is successful then it can return the
  view-shift transitioning $\All vs. P(vs) \vsW (v :: vs) * Q)$ or it
  fails, implying that some thread has already successfully executed a
  take. If the latter is the case, the thread can return the $Q$ that
  must be stored in $\textlog{stages}$ which must be in the disjunct
  $\ell \mapsto 1 * Q$.
\end{enumerate}
With this we can specify the invariant enforced on offers.
\[
  \textlog{is\_offer}_\gamma(\iota, v) \triangleq
  \Exists v' l. v = (v', l) * \Exists \iota'. \iota \disjoint \iota' *
  \knowInv{\iota'}{\textlog{stages}_\gamma(l)}
\]
The only difference worth noting is the inclusion of this
$\iota' \disjoint \iota$. This is purely an artifact of the particular
approach the proof of the specification has taken. At various points,
it will be necessary to simultaneously open the invariant containing
knowledge of the stack as well as the invariant containing what stage
an offer is in. For instance, this is how {\tt take\_offer} will work
so that it can actually apply the view-shift to the current state of
the stack. Opening two invariants at the same time, however, is only
possible if the invariants are in fact disjoint. All that this premise
does is manually record the fact that this is the case. Later on the
$\iota$ parameterizing $\textlog{is\_offer}$ will be the same as the
$\iota$ governing the stack invariant.

This change also propagates through the $\textlog{is\_mailbox}$
invariant since this is after all where $\textlog{is\_offer}$
occurs. Apart from this plumbing though the invariant is unchanged
from the prior section.
\[
  \textlog{is\_mailbox}(\iota, v) \triangleq
  \Exists \ell. v = \ell *
  \ell \mapsto \textlog{None} \mathrel{\vee}
  \Exists v' \gamma. l \mapsto \textlog{Some}(v') * \textlog{is\_offer}_\gamma(\iota, v')
\]
Rather than proceeding to formalize the specifications of mailboxes
and offers, it is simpler to proceed directly to formalizing the
stack data structure itself. The invariant governing a stack is
unchanged from the previous section.
\begin{align*}
  \textlog{is\_stack}([], v) &\triangleq v = \textlog{None}\\
  \textlog{is\_stack}(x :: L, v) &\triangleq v =
  \Exists t. v = \textlog{Some}(x, t) * \textlog{is\_stack}(L, t)\\
  \textlog{stack\_inv}(v) &\triangleq
  \Exists \ell, v', L. v = \ell * \ell \mapsto v' * P(L) * \textlog{is\_stack}(L, v')
\end{align*}
The final specification is again unchanged. For brevity, the proof
(which is largely a combination of the two previously explained ones)
is only sketched.
\begin{thm}
  The following specification holds.
  \begin{align*}
    \forall \Phi.\qquad&\\
    (\All f_1 f_2.& \\
    (&(\All vs. P(v :: vs) \vsW Q_1(v) * P(vs)) \wand \\
                       &(\All vs. P([]) \vsW Q_2 * P([])) \wand\\
                       &\wpre{f_1()}{v. v = \textlog{None} * Q_1 \mathrel{\vee} \Exists v'. v = \textlog{Some}(v') * Q_1(v')}\\
    \wand \quad (&\All v. (\All vs. P(vs) \vsW Q * P(v :: vs)) \wand \wpre{\text{\tt f2(v)}}{v. Q}) \\
    \wand \quad \phantom{(}& \Phi(f_1, f_2))\\
    \wand P([&]) \\
    \wand \wpre{&\textlog{stack}()}{\Phi}
  \end{align*}
\end{thm}
\begin{proof}
  The beginning of the proof is a straightforward combination of the
  previous two proofs, allocating an invariant for the mailbox and the
  mutable cell containing the stack: $\knowInv{\iota}{\textlog{stack\_inv}(r)}$
  and $\knowInv{\iota'}{\textlog{is\_mailbox}(r', \iota)}$. Notice that we pass
  the $\iota$ from the stack invariant to the mailbox invariant as
  discussed above.

  Having done this, we apply the assumption that
  \begin{align*}
    \All f_1, f_2.&\\
    (&(\All vs. P(v :: vs) \vsW Q_1(v) * P(vs)) \wand \\
     &(\All vs. P([]) \vsW Q_2 * P([])) \wand\\
     &\wpre{f_1()}{v. v = \textlog{None} * Q_1 \mathrel{\vee} \Exists v'. v = \textlog{Some}(v') * Q_1(v')})\\
    \wand \quad (&\All v. (\All vs. P(vs) \vsW Q * P(v :: vs)) \wand \wpre{f_2(v)}{v. Q})\\
    \wand \quad \phantom{(}& \Phi(f_1, f_2)
  \end{align*}
  This leaves us with two separate verifications for push and
  pop. These are largely the same as the previous proof. The only
  difference is each proof is proceeded by manual manipulations of
  the mailboxes and offers which in turn are simply inlined versions
  of the proof from the prior sections.

  We will consider the case for {\tt push}. The goal we must prove is
  then
  \[
    \All v. (\All vs. P(vs) \vsW Q * P(v :: vs))
    \wand \wpre{\text{\tt push(v)}}{v. Q}
  \]
  Therefore, let us suppose that $\All vs. P(vs) \vsW Q * P(v :: vs)$
  and that we have some value $v$. It as at this point that we apply
  L\"ob induction to add an induction hypothesis of
  $\later \wpre{\text{\tt push(v)}}{v. Q}$ to our context. We then
  simplify our goal using $\beta$ and bind to
  \[
    \wpre{\text{\tt put v}}{v'. \wpre{\text{\tt match v' with ...}}{Q}}
  \]
  Since we have no specification yet formalized for {\tt push}, we
  proceed to unfold it and begin to work with the internal
  implementation of {\tt put}.
  \[
    \wpre{\text{\tt let off = mk\_offer v in ...}}{v'. \wpre{\text{\tt match v' with ...}}{Q}}
  \]
  We can simplify this again and apply bind
  \[
    \wpre{\text{\tt (v, ref 0)}}
    {v'. \wpre{\text{\tt let off = v' in ...}}{v'. \wpre{\text{\tt match v' with ...}}{Q}}}
  \]
  We can then apply the rule for allocation, so suppose that we have
  some $\ell$ and $\ell \mapsto 0$. We then must show that
  \[
    \wpre{\text{\tt r' := Some (v, l); revoke\_offer (v, l)}}
    {v'. \wpre{\text{\tt match v' with ...}}{Q}}
  \]
  We may then open the invariant
  $\knowInv{\iota'}{\textlog{is\_mailbox}(r', \iota)}$ and apply the
  replace the previous value stored in $r'$ with {\tt (v, l)} and
  reestablish the invariant
  $\knowInv{\iota'}{\textlog{is\_mailbox}(r', \iota)}$. In order to do
  this, we must use $\All vs. P(vs) \vsW Q * P(v :: vs)$ and allocate
  a new invariant $\knowInv{\iota''}{\textlog{stages}_\gamma(l)}$ for
  some $\gamma$ so that we also have $\ownGhost{\gamma}{\exinj(())}$.
  Thus, our goal is now
  \[
    \wpre{\text{\tt revoke\_offer (v, l)}}{v'. \wpre{\text{\tt match v' with ...}}{Q}}
  \]
  If we apply $\beta$ rules we then end up with
  \[
    \wpre{(\text{\tt if cas l 0 2 then Some(v) else None})}{v'. \wpre{\text{\tt match v' with ...}}{Q}}
  \]
  Let us then open up the invariant
  $\knowInv{\iota''}{\textlog{stages}(l)}$. This tells us that there
  \[
    (\ell \mapsto 0 * (\All vs. P(vs) \vsW P (v :: vs) * Q))
    \mathrel{\vee} (\ell \mapsto 1 * Q)
    \mathrel{\vee} \ell \mapsto 1 * \ownGhost{\gamma}{\exinj(())}
    \mathrel{\vee} (\ell \mapsto 2 * \ownGhost{\gamma}{\exinj(())})
  \]
  Let us perform case analysis on these three possible results. We can
  immediately dismiss the last two cases since we have assumed to own
  $\ownGhost{\gamma}{\exinj(())}$ which means that the
  $\ownGhost{\gamma}{\exinj(())}$ coming from this invariant would
  produce a contradiction.

  In the second case, the CAS fails so we are left with $\ell \mapsto
  1$, $\ownGhost{\gamma}{\exinj(())}$, and $Q$. Let us reestablish our
  invariant using the $\ell \mapsto 1$ and
  $\ownGhost{\gamma}{\exinj(())}$ using the second branch. Our goal is then
  \[
    \wpre{(\text{\tt None})}{v'. \wpre{\text{\tt match v' with ...}}{Q}}
  \]
  and simplifying this gives us the obligation
  \[
    \wpre{\text{\tt None}}{Q}
  \]
  but this is trivial to discharge using our assumption of $Q$ that we
  got from the invariant. This case is essentially the position we
  would be in if the side-channel is accepted.


  In the first case, therefore, we must consider what happens if the
  side-channel is not used. Clearly the CAS is successful
  and we can reestablish this invariant by providing
  $\ownGhost{\gamma}{\exinj(())}$ and choosing the last disjunct. In
  this case, we have the assumption $\All vs. P(vs) \vsW P (v :: vs) * Q$
  and the goal
  \[
    \wpre{(\text{\tt Some(v)})}{v'. \wpre{\text{\tt match v' with ...}}{Q}}
  \]
  Therefore, we can apply $\beta$ rules to reduce this to
  \[
    \wpre{\text{\tt let x = !r in let y = Some (v, x) in ...}}{Q}
  \]
  In order to reduce this, we must open the invariant
  $\knowInv{\iota}{\textlog{stack\_inv}(r)}$ which tells us, amongst
  other things, that $r = \ell'$ for some $\ell'$ and that $\ell'
  \mapsto s$ for some value $s$. We can then step our program and
  reestablish the invariant (stepping does not effect the context at
  all) to
  \[
    \wpre{\text{\tt let x = s in let y = Some (v, x) in ...}}{Q}
  \]
  We can now repeatedly simplify our goal to
  \[
    \wpre{\text{\tt if cas r s (Some (v, s)) then () else push v}}{Q}
  \]
  In order to simplify this, we must once again open up the invariant
  $\knowInv{\iota}{\textlog{stack\_inv}(r)}$. This tells us that for
  some $s'$, $L$ and $\ell'$ that
  \[
    r = \ell' * \ell' \mapsto s' * P(L) * \textlog{is\_stack}(L, s')
  \]
  holds. Let us then case on whether or not $s' = s$.

  If it does, then the CAS succeeds, therefore we have
  $\ell' \mapsto \textlog{Some}(v, s)$. Simple logic gives us that
  $\textlog{is\_stack}(L, s')$ implies that
  $\textlog{is\_stack}(v :: L, \textlog{Some}(v, s))$
  holds. In order to reestablish our invariant we must show that $\vs
  \textlog{stack\_inv}(r)$ holds. However, since we have the
  implication $P(L) \vsW P(v :: L)$ it is sufficient to show that
  \[
    r = \ell' * \ell' \mapsto \textlog{Some}(v, s) * P(L)
    * \textlog{is\_stack}(v :: L, \textlog{Some}(v, s))
  \]
  however we have precisely these assumptions so we're done. This
  leaves us with the goal $\wpre{()}{Q}$ which is discharged
  immediately with our assumption of $Q$.


  If these are not equal, then the CAS fails so it's trivial to
  reestablish the invariant. Our goal is then $\wpre{\text{\tt push v}}{Q}$
  but for this we just apply our IH and we're done.
\end{proof}

The Coq formalization of this specification and proof may be found in
{\tt concurrent\_stack4.v}.

%%% Local Variables:
%%% mode: latex
%%% TeX-master: "main"
%%% End:

\section{Conclusion}

In this case study we have examined several different incarnations of
formalizations of concurrent stacks in Iris. This provides evidence
for Iris being an expressive and flexible program logic. Several of
Iris's features were necessary to even express the desired
specifications.
\begin{itemize}
\item Impredicative invariants were needed in order to have the
  invariant contain $P$, the arbitrary predicate all the
  specifications where parameterized by.
\item Higher-order specifications in order to describe the
  \emph{closure-returning} pattern that mailboxes and stacks made use
  of.
\item View-shifts in order to express linearization points.
\end{itemize}
Furthermore, the encoding of state transition systems as a simple
proposition using ghost state demonstrates how simple CMRAs are
sufficient to encode complex logical structures for expressing the
structure of our program.

All of these specifications where heavily inspired
by~\citet{Clausen:2017} which provided a similar verification of
hash-tables in Iris. Future work in this direction would be to mimic
this work and drive towards more compositional verification of
concurrent stacks. Ideally, the proof could be decomposed in the same
that the proof of the bag specification is: respecting abstraction
boundaries of APIs and relying purely on the specifications. More
generally, there is still a great deal of engineering as well as
theoretical to work in specifying sophisticated data-structures in a
useful but still provable way.

%%% Local Variables:
%%% mode: latex
%%% TeX-master: "main"
%%% End:


\bibliographystyle{plainnat}
\nocite{*}
\bibliography{iris}{}
\end{document}
