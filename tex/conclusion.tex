\section{Conclusion}

In this case study we have examined several different incarnations of
formalizations of concurrent stacks in Iris. This provides evidence
for Iris being an expressive and flexible program logic. Several of
Iris's features were necessary to even express the desired
specifications.
\begin{itemize}
\item Impredicative invariants were needed in order to have the
  invariant contain $P$, the arbitrary predicate all the
  specifications where parameterized by.
\item Higher-order specifications in order to describe the
  \emph{closure-returning} pattern that mailboxes and stacks made use
  of.
\item View-shifts in order to express linearization points.
\end{itemize}
Furthermore, the encoding of state transition systems as a simple
proposition using ghost state demonstrates how simple CMRAs are
sufficient to encode complex logical structures for expressing the
structure of our program.

All of these specifications where heavily inspired
by~\citet{Clausen:2017} which provided a similar verification of
hash-tables in Iris. Future work in this direction would be to mimic
this work and drive towards more compositional verification of
concurrent stacks. Ideally, the proof could be decomposed in the same
that the proof of the bag specification is: respecting abstraction
boundaries of APIs and relying purely on the specifications. More
generally, there is still a great deal of engineering as well as
theoretical to work in specifying sophisticated data-structures in a
useful but still provable way.

%%% Local Variables:
%%% mode: latex
%%% TeX-master: "main"
%%% End:
